\begin{pa} \label{PA:11.6} Recall the standard parameterization of the unit circle that is given by
\[x(t) = \cos(t) \ \ \ \ \text{ and } \ \ \ \ y(t) = \sin(t),\]
where $0 \le t \le 2\pi$.
    \ba
        \item Determine a parameterization of the circle of radius 1 in $\R^3$ that has its center at $(0,0,1)$ and lies in the plane $z=1$.

        \item Determine a parameterization of the circle of radius 1 in 3-space that has its center at $(0,0,-1)$ and lies in the plane $z=-1$.

        \item Determine a parameterization of the circle of radius 1 in 3-space that has its center at $(0,0,5)$ and lies in the plane $z=5$.

        \item Taking into account your responses in (a), (b), and (c), describe the graph that results from the set of parametric equations
        \[x(s,t) = \cos(t), \ \ \ \ y(s,t) = \sin(t), \ \ \ \ \text{ and } \ \ \ \ z(s,t) = s,\]
        where $0 \le t \le 2\pi$ and $-5 \le s \le 5$.         Explain your thinking.


    \item Just as a cylinder can be viewed as a ``stack'' of circles of constant radius, a cone can be viewed as a stack of circles with varying radius.  Modify the parametrizations of the circles above in order to construct the parameterization of a cone whose vertex lies at the origin, whose base radius is 4, and whose height is 3, where the base of the cone lies in the plane $z = 3$. Use appropriate technology\footnote{e.g., \url{http://www.flashandmath.com/mathlets/multicalc/paramrec/surf_graph_rectan.html}} to plot the parametric equations you develop. (Hint: The cross sections parallel to the $xz$ plane are circles, with the radii varying linearly as $z$ increases.)


    \ea
 
\end{pa} 
  
\begin{activitySolution}

    \ba
        \item A parameterization of the circle of radius 1 in $\R^3$ that has its center at $(0,0,1)$ and lies in the plane $z=1$ is
\[x(t) = \cos(t), \ y(t) = \sin(t), \ \text{ and } \ z(t) = 1.\]

        \item A parameterization of the circle of radius 1 in 3-space that has its center at $(0,0,-1)$ and lies in the plane $z=-1$ is
\[x(t) = \cos(t), \ y(t) = \sin(t), \ \text{ and } \ z(t) = -1.\]

        \item A parameterization of the circle of radius 1 in 3-space that has its center at $(0,0,5)$ and lies in the plane $z=5$.
\[x(t) = \cos(t), \ y(t) = \sin(t), \ \text{ and } \ z(t) = 5.\]

        \item This set of parametric equations would describe a cylinder centered around the $z$-axis, with a radius of $1$, that extended from $z=-5$ to $z=5$. The parameter $s$ in $z(t)$ allows the $z$ value to vary independently of the parameter $t$, which traces out the circles at each cross-sectional level of the cylinder.

    \item If we set the vertex of the cone at the origin and orient the cone to open upwards, then the cross sections parallel to the $xz$ plane are again circles, with the radii varying linearly as $z$ varies. So the parameterization defined by 
\[\vr(s,t) = t \cos(s) \vi + t \sin(s) \vj + t \vk\]
for $t \geq 0$ will gives us a cone.

If we want a cone with base radius 4 and height 3, we can modify the equations a bit. We need the radii to increase linearly so that when $z$ is 3 the radius of the cross section parallel to the $xy$ plane is 4. That means we need the radii to be $\frac{4}{3}z$ as $z$ increases from 0 to 3. So a parameterization that will gives us this cone is defined by
\[\vr(s,t) = \frac{4}{3}t \cos(s) \vi + \frac{4}{3} t \sin(s) \vj + t \vk\]
for $t$ from 0 to 3.

    \ea


\end{activitySolution}    

\afterpa 