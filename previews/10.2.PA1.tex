\begin{pa} \label{PA:10.2} An object in motion can do work. As an example, a wrecking ball is effective because a massive object moving at a large velocity possesses a significant amount of energy that can be imparted on collision. The energy that an object possesses due to its motion is called \emph{kinetic energy}. Kinetic energy is defined as the work required to accelerate a body of a given mass $m$ from rest to its current velocity $v$, so kinetic energy is a function of the two variables $m$ and $v$. More specifically, the kinetic energy $f$ of an object of mass $m$ and velocity $v$ is given by
\[f(m,v) = \frac{1}{2}mv^2,\]
where $f$ is measured in Joules if the mass $m$ is in kilograms and the velocity $v$ is in meters per second.
    \ba
    \item Fix the mass of the object to be 2 kilograms. Write $f$ as a function of the velocity alone. Then find the derivative of $f$ with respect to the velocity (keeping the mass fixed at 2 kg) when the velocity is 3 meters per second. Include the units. Explain in detail what this tells us about kinetic energy.

\begin{comment}

If the mass $m$ is fixed at 2 kilograms, then
\[f = f(2,v) = v^2\]
is a function of $v$ alone. So
\[\frac{d}{dv} f(2,v) = 2v,\]
and
\[\frac{d}{dv} f(2,v)\bigm|_{v=3} = 6 \ \frac{\text{Joules}}{\text{meters/second}}.\]
This last calculation tells us that if we increase the velocity by 1 meter per second when the velocity is 3 meters per second and the mass is 2 kilograms, then the kinetic energy increases by approximately 6 Joules.

\end{comment}

    \item Now let the mass vary but keep the velocity constant at 5 meters per second. Write $f$ as a function of the mass alone. Then find the derivative of $f$ with respect to the mass (keeping the velocity fixed at 5 meters per second) when the mass is 2 kilograms. Include the units. Explain in detail what this tells us about kinetic energy.

\begin{comment}

If the velocity $v$ is fixed at 5 meters per second, then
\[f = f(m,5) = 12.5m\]
is a function of $m$ alone. So
\[\frac{d}{dm} f(m,5) = 12.5,\]
and
\[\frac{d}{dv} f(m,5)\bigm|_{m=2} = 12.5 \ \frac{\text{Joules}}{\text{kilogram}}.\]
This last calculation tells us that if we increase the mass by 1 kilogram when the mass is 2 kilograms and the velocity is 5 meters per second, then the kinetic energy increases by approximately 12.5 Joules.

\end{comment}

    \ea


\end{pa} \afterpa 