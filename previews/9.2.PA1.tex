\begin{pa} \label{PA:9.2}
After working out, Sarah and John leave the Recreation Center on the Grand Valley State University Allendale campus (a map of which is given in Figure \ref{F:9.2.GVSU}) to go to their next classes.\footnote{GVSU campus map from \url{http://www.gvsu.edu/homepage/files/pdf/maps/allendale.pdf}, used with permission from GVSU, credit to illustrator Chris Bessert.}  Suppose we record Sarah's movement on the map in a pair $\langle x, y \rangle$ (we will call this pair a \emph{vector}), where $x$ is the horizontal distance (in feet) she moves (with east as the positive direction) and $y$ as the vertical distance (in feet) she moves (with north as the positive direction). We do the same for John. Throughout, use the legend to estimate your responses as best you can. 

    \ba
   \item What is the vector $\vv_1 = \langle x , y \rangle$ that describes Sarah's movement if she walks directly in a straight line path from the Recreation Center to the entrance at the northwest end of Mackinac Hall? (Assume a straight line path, even if there are buildings in the way.) Explain how you found this vector. What is the total distance in feet between the Recreation Center and the entrance to Mackinac Hall?  Measure the number of feet directly and then explain how to calculate this distance in terms of $x$ and $y$.



    \item What is the vector $\vv_2 = \langle x , y \rangle$ that describes John's change in position if he walks directly from the Recreation Center to Au Sable Hall? How many feet are there between Recreation Center to Au Sable Hall in terms of $x$ and $y$?



    \item What is the vector $\vv_3 = \langle x , y \rangle$ that describes the change in position if John walks directly from Au Sable Hall to the northwest entrance of Mackinac Hall to meet up with Sarah after class? What relationship do you see among the vectors $\vv_1$, $\vv_2$, and $\vv_3$? Explain why this relationship should hold.



	\ea


\end{pa} 

\begin{activitySolution}
    \ba
   \item Using a marked straightedge, the horizontal distance from the Recreation Center to the entrance at the northwest end of Mackinac Hall is approximately 750 feet, while the vertical distance is about 420 feet. Since Sarah traveled east and north, the vector describing her movement is ${\vv_1 = \langle 750, 420 \rangle}$. The straight line distance from the Recreation Center to the entrance at the northwest end of Mackinac Hall is measured with straightedge as approximately 875 feet, and this can be found using the Pythagorean Theorem on the right triangle with $x$ and $y$ as legs and the distance from the Recreation Center to the entrance at the northwest end of Mackinac Hall as the hypotenuse. This gives the straight line distance from the Recreation Center to the entrance at the northwest end of Mackinac Hall as
\[\sqrt{(750)^2 + (420)^2} \approx 859.6 \text{ feet}.\]
Of course, there is error here in all of the measurements.

    \item Using a marked straightedge, the horizontal distance from the Recreation Center to Au Sable Hall is approximately 1200 feet, while the vertical distance is about 1100 feet. Since John traveled east and south, the vector describing his movement is ${\vv_2 = \langle 1200, -1100 \rangle}$. John walked approximately
\[\sqrt{1200^2 + (-1100)^2} \approx 1627.88 \text{ feet}.\]


    \item Using a marked straightedge, the horizontal distance from Au Sable Hall to the northwest entrance of Mackinac is approximately 450 feet, while the vertical distance is about 1520 feet. Since John traveled west and north, the vector describing this movement is ${\vv_3 = \langle -450, 15205 \rangle}$. The horizontal distance traveled from Au Sable Hall to the northwest entrance of Mackinac is the difference of the distances from the Recreation Center to Au Sable Hall and from the Recreation Center to the northwest entrance of Mackinac (this is because John traveled in opposite horizontal directions), while the vertical distance from Au Sable Hall to the northwest entrance of Mackinac is the sum of the vertical distance from the Recreation Center to the northwest entrance of Mackinac and the vertical distance from the Recreation Center to Au Sable Hall (since John travels in the vertical direction). So, in essence,
\[\vv_3 = \vv_1 - \vv_2.\]


	\ea

\end{activitySolution}

\afterpa 