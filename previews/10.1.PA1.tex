\begin{pa} \label{PA:10.1} In this section we study limits of
  functions of several variables, with a focus on limits of functions
  of two variables. 

\ba
    \item Review the definition of the limit of a function $f$ of a
      single variable $x$ from first semester calculus and correctly
      complete the following sentence. 


\begin{quote} A function $f = f(x)$ has a limit at the point where
  $x=a$ if ... . 
\end{quote}


\begin{comment}

In single variable calculus we stated that a function $f$ of the
single variable $x$ has a limit $L$ at the point where $x = a$ if we
can make all of the values of $f(x)$ as close as we want to $L$ by
choosing $x$ as close to $a$ (but not equal to $a$) as we need. 

\end{comment}

    \item Now let $f$ be the function of the variables $x$ and $y$
      defined by $f(x,y) = \frac{x^2+y^2}{x+y}$. 
        \begin{enumerate}[i.]
        \item Let $y$ be held constant at $y=1$. Write a formula for
          $f$ then as a function of $x$ alone on this trace. Does this
          single variable function $f$ have a limit at $x=2$? If yes,
          explain why and find the limit. If no, explain why not. 

\begin{comment}

If we hold $y$ constant at $y=1$, then $f(x,1)$ is a function of $x$
alone. More specifically, 
\[f(x,1) = \frac{x^2+1}{x+1}.\]
Since $f(x,1)$ is a quotient of continuous functions and the denominator is defined at $x=2$, we have
\[\lim_{x \to 2} f(x,1) = \frac{2^1+1}{2+1} = \frac{5}{3}.\]

\end{comment}

        \item Now let $y$ vary and hold $x$ constant at $x=0$. Write a
          formula for $f$ then as a function of $y$ alone on this
          trace. Does this single variable function $f$ have a limit
          at $y=0$? If yes, explain why and find the limit. If no,
          explain why not. 

\begin{comment}

If we hold $x$ constant at $x=0$, then $f(0,y)$ is a function of $y$
alone. More specifically, 
\[f(0,y) = \frac{y^2}{y}.\]
Since we don't let $y$ actually equal 0 in the limit, we have
\[\lim_{y \to 0} f(0,y) = \lim_{y \to 0} \frac{y^2}{y} = \lim_{y \to 0} y = 0.\]

\end{comment}

        \end{enumerate}

    \item Now let $f$ be any function of two variables $x$ and
      $y$. Use the idea of limit as reviewed in part (a) to complete
      the following statement of what it means for $f = f(x,y)$ to
      have a limit $L$ at a point $(x, y) = (a, b)$. 

\begin{quote} A function $f$ of the two variables $x$ and $y$ has a
  limit $L$ at the point where $(x, y) = (a, b)$ if $\ldots$ 
\end{quote}

\begin{comment}

A function $f = f(x,y)$ of the independent variables $x$ and $y$ has a
limit $L$ at the point where $(x,y) = (a,b)$ if we can make all of the
values of $f(x,y)$ as close as we want to $L$ by choosing $(x,y)$ as
close to $(a,b)$ (but not equal to $(a,b)$) as we need. 

\end{comment}

    \ea


\end{pa} \afterpa 