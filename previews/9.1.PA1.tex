\begin{pa} \label{PA:9.1} When people buy a large ticket item like a
  car or a house, they often take out a loan to make the purchase. The
  loan is paid back in monthly installments until the entire amount of
  the loan, plus interest, is paid. The monthly payment that the
  borrower has to make depends on the amount $P$ of money borrowed
  (called the principal), the duration $t$ of the loan in years, and
  the interest rate $r$. For example, if we borrow \$18,000 to buy a
  car, the monthly payment $M$ that we need to make to pay off the
  loan is given by the formula
\[M = \frac{1500r}{1-\frac{1}{\left(1+\frac{r}{12}\right)^{12t}}}.\]
The variables $r$ and $t$ are independent of each other, so using functional notation we write
\[M(r,t) = \frac{1500r}{1-\frac{1}{\left(1+\frac{r}{12}\right)^{12t}}}.\]
    \ba
    \item Find the monthly payments on this loan if the interest rate is 6\% and the duration of the loan is 5 years.

    \item Evaluate $M(0.05, 4)$. Explain in words what this calculation represents.

\item Now consider only loans where the interest rate is
  5\%. Calculate the monthly payments as indicated in Table
  \ref{T:9.1.Payments_1}. Round payments to the nearest penny.
    \begin{table}[ht]
    \begin{center}
    \renewcommand{\arraystretch}{1.5}
    \begin{tabular}{|c|c|c|c|c|c|} \hline
    Duration (in years)  &\ 2 \    & \ 3 \     & \ 4 \     & \ 5 \    & \ 6 \  \\ \hline
    Monthly payments (dollars)    & \hspace*{0.55in}  &
    \hspace*{0.55in} & \hspace*{0.55in} & \hspace*{0.55in} &
    \hspace*{0.55in} \\ \hline  
    \end{tabular}
    \caption{Monthly payments at an interest rate of 5\%.}
    \label{T:9.1.Payments_1}
    \end{center}
    \end{table}

    \item Now consider only loans where the duration is 3 years. Calculate the monthly payments as indicated in Table \ref{T:9.1.Payments_2}. Round payments to the nearest penny.
    \begin{table}[ht]
    \begin{center}
    \renewcommand{\arraystretch}{1.5}
    \begin{tabular}{|c|c|c|c|c|c|} \hline
    Interest rate  &\ 0.03 \    & \ 0.05 \     & \ 0.07 \     & \ 0.09 \    & \ 0.11 \  \\ \hline
    Monthly payments (dollars)    & \hspace*{0.55in} & \hspace*{0.55in} & \hspace*{0.55in} & \hspace*{0.55in} & \hspace*{0.55in}\\ \hline
    \end{tabular}
    \caption{Monthly payments over three years.}
    \label{T:9.1.Payments_2}
    \end{center}
    \end{table}

    \item Describe as best you can the combinations of interest rates and durations of loans that result in a monthly payment of \$200.


    \ea
\end{pa} 

\begin{activitySolution}
    \ba
    \item We simply calculate $M$ when $r=0.06$ and $t=5$ to obtain (rounded to the nearest penny)
\[M(0.06,5) = 347.99 \text{ dollars}.\]

    \item We evaluate $M$ when $r=0.05$ and $t=4$ to obtain (rounded to the nearest penny)
\[M(0.05,4) = 414.53 \text{ dollars}.\]
This tells us that the monthly payment on a loan of \$18,000 at 5\% interest for 4 years is \$414.53.

\item We complete the table using 5\% as our interest rate with varying durations as shown below.

%\begin{table}[ht]
    \begin{center}
    \renewcommand{\arraystretch}{1.5}
    \begin{tabular}{|c|c|c|c|c|c|} \hline
    Duration (in years)  &\ 2 \    & \ 3 \     & \ 4 \     & \ 5 \    & \ 6 \  \\ \hline
    Monthly payments (dollars)    & 789.68      & 539.48      & 414.53      & 339.68      & 289.89 \\ \hline
    \end{tabular}
%    \caption{Monthly payments at an interest rate of 5\%.}
%    \label{T:9.1.Payments_1_s}
    \end{center}
%    \end{table}

    \item We complete the table using 3 years as our duration with interest rates as shown in the table below. 

%   \begin{table}[ht]
    \begin{center}
    \renewcommand{\arraystretch}{1.5}
    \begin{tabular}{|c|c|c|c|c|c|} \hline
    Interest rate  &\ 0.03 \    & \ 0.05 \     & \ 0.07 \     & \ 0.09 \    & \ 0.11 \  \\ \hline
    Monthly payments (dollars)    & 523.46       & 539.48      & 555.79      & 572.40      & 589.30 \\ \hline
    \end{tabular}
%    \caption{Monthly payments over three years.}
%    \label{T:9.1.Payments_2_s}
    \end{center}
%    \end{table}

    \item We set $M(r,t)$ equal to 200 and determine a relationship between $r$ and $t$:
\begin{align}
\frac{1500r}{1-\frac{1}{\left(1+\frac{r}{12}\right)^{12t}}} &= 200 \notag \\
7.5r &= 1-\frac{1}{\left(1+\frac{r}{12}\right)^{12t}} \notag \\
1-7.5r &= \frac{1}{\left(1+\frac{r}{12}\right)^{12t}} \notag \\
\left(1+\frac{r}{12}\right)^{12t} &= \frac{1}{1-7.5r} \notag \\
12t \ln\left(1+\frac{r}{12}\right) &= \ln\left(\frac{1}{1-7.5r}\right) \notag \\
t &= \frac{1}{12}\frac{-\ln\left(1-7.5r\right)}{\ln\left(1+\frac{r}{12}\right)}. \label{eq:monthly_payments} 
\end{align}
So the duration of the loan and the corresponding interest rates that lead to a monthly payment of \$200 are given by the equation \ref{eq:monthly_payments}.
    \ea
\end{activitySolution}
\afterpa 