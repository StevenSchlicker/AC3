\begin{pa} \label{PA:10.7} 
Let $z = f(x,y)$ be a differentiable function, and suppose that at the point $(x_0, y_0)$, $f$ achieves a local maximum.  That is, the value of $f(x_0,y_0)$ is greater than the value of $f(x,y)$ for all $(x,y)$ nearby $(x_0,y_0)$.  You might find it helpful to sketch a rough picture of a possible function $f$ that has this property.

  \ba 
  \item If we consider the trace given by
    holding $y=y_0$ constant, then the single-variable function $f(x, y_0)$ must have a local maximum at
    $x_0$.  What does this say about the value of the partial derivative
    $f_x(x_0,y_0)$? 
     
  \item In the same way, the trace given by holding $x=x_0$ constant
    has a local maximum at $y=y_0$.  What does this say about the value of the
    partial derivative $f_y(x_0,y_0)$?
    
  \item What may we now conclude about the gradient $\nabla f(x_0,y_0)$ at the
    local maximum?  How is this consistent with the statement ``$f$
    increases most rapidly in the direction $\nabla f(x_0,y_0)$?''
    
  \item How will the tangent plane to the surface $z = f(x,y)$ appear at the point $(x_0, y_0, f(x_0,y_0))$?
  
  \item By first computing the partial derivatives, find any points at which the function $f(x,y) = 2x -   x^2 - (y + 2)^2$ may have a local maximum.
  \ea
\end{pa} 

\begin{activitySolution} 
   \ba 
  \item Since $f$ is a differentiable function, its derivative $f_x$ in the $x$-direction must be 0 at a local maximum. In other words, $f_x(x_0,y_0) = 0$. 
     
  \item Since $f$ is a differentiable function, its derivative $f_y$ in the $y$-direction must be 0 at a local maximum. In other words, $f_y(x_0,y_0) = 0$.
    
  \item Since $\nabla f(x_0,y_0) = \langle f_x(x_0,y_0), f_y(x_0,y_0) \rangle$, we must have $\nabla f(x_0,y_0) = \vzero$. Since the function has a local maximum at the point $(x_0,y_0)$, the rate of change of $f$ in any direction at $(x_0,y_0)$ is 0. 
    
  \item With both $f_x(x_0,y_0)=0$ and $f_y(x_0,y_0) = 0$, the tangent plane to the surface $z = f(x,y)$ at the point $(x_0, y_0, f(x_0,y_0))$ will have the form $z = f(x_0,y_0)$. So the tangent plane will be parallel to the $x$-$y$ plane. 
  
  \item We have 
\[f_x(x,y) = 2-2x \ \text{ and } \ f_y(x,y) = 2(y+2),\]
so $\nabla f = \vzero$ at the point $(1,-2)$. This is the only point at which the function $f(x,y) = 2x -   x^2 - (y + 2)^2$ may have a local maximum.
  \ea
\end{activitySolution}

\afterpa 