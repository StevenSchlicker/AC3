\begin{pa} \label{PA:11.5}  The coordinates of a point determine its location. In particular, the rectangular coordinates of a point $P$ are given by an ordered pair $(x,y)$, where $x$ is the (signed) distance the point lies from the $y$-axis to $P$ and $y$ is the (signed) distance the point lies from the $x$-axis to $P$.  In polar coordinates, we locate the point by considering the distance the point lies from the origin, $(0,0)$, and the angle the line segment from the origin to $P$ forms with the positive $x$-axis.
    \ba
    \item Determine the rectangular coordinates of the following points:
        \begin{enumerate}[i.]
        \item The point $P$ that lies 1 unit from the origin on the positive $x$-axis.


        \item The point $Q$ that lies 2 units from the origin and such that $\overline{OQ}$ makes an angle of $\frac{\pi}{2}$ with the positive $x$-axis, where $O$ is the origin, $(0,0)$.


        \item The point $R$ that lies 3 units from the origin such that $\overline{OR}$ makes an angle of $\frac{2\pi}{3}$ with the positive $x$-axis.

        \end{enumerate}

    \item Part (a) indicates that the two pieces of information completely determine the location of a point:  either the traditional $(x,y)$ coordinates, or alternately, the distance $r$ from the point to the origin along with the angle $\theta$ that the line through the origin and the point makes with the positive $x$-axis. We write ``$(r, \theta)$'' to denote the point's location in its polar coordinate representation. Find polar coordinates for the points with the given rectangular coordinates.
        \begin{enumerate}[i.]
        \item $(0,-1)$  \hspace{1.0in} ii.  $(-2,0)$  \hspace{1.0in} iii.  $(-1,1)$

        \end{enumerate}

	\item For each of the following points whose coordinates are given in polar form, determine the rectangular coordinates of the point.
       \begin{enumerate}[i.]
          \item $(5, \frac{\pi}{4})$         \hspace{1.0in} ii. $(2, \frac{5\pi}{6})$   \hspace{1.0in} iii. $(\sqrt{3}, \frac{5\pi}{3})$
       \end{enumerate}

    \ea

\end{pa} 

\begin{activitySolution} 
   \ba
    \item Determine the rectangular coordinates of the following points:
        \begin{enumerate}[i.]
        \item This point has $y$ coordinate 0, so the $x$ coordinate must be 1 and the point is $(1,0)$.

        \item The angle of $\frac{\pi}{2}$ with the positive $x$-axis places the point on the $y$-axis. The point on the $y$-axis a distance 2 from the origin is the point $(0,2)$.

        \item Draw a right triangle from the origin to $R$ to the point on the $x$-axis below $R$. The acute angle at the origin has measure $\frac{\pi}{3}$. So $x = 3\cos\left(\frac{2\pi}{3}\right) = -\frac{3}{2}$ and $y = 3\sin\left(\frac{2\pi}{3}\right) = \frac{3\sqrt{3}}{2}$. So this point has rectangular coordinates $\left(-\frac{3}{2}, \frac{3\sqrt{3}}{2}\right)$.

        \end{enumerate}

    \item Part (a) indicates that the two pieces of information completely determine the location of a point:  either the traditional $(x,y)$ coordinates, or alternately, the distance $r$ from the point to the origin along with the angle $\theta$ that the line through the origin and the point makes with the positive $x$-axis. We write ``$(r, \theta)$'' to denote the point's location in its polar coordinate representation. Find polar coordinates for the points with the given rectangular coordinates.
        \begin{enumerate}[i.]
        \item The distance from the point with rectangular coordinates $(0,-1)$ to the origin is 1, and the angle the line through the origin and this point makes with the positive $x$-axis is $\frac{\pi}{2}$, so the polar coordinate representation of this point is $\left(1,\frac{\pi}{2}\right)$.

	\item The distance from the point with rectangular coordinates $(-2,0)$ to the origin is 2, and the angle the line through the origin and this point makes with the positive $x$-axis is $\pi$, so the polar coordinate representation of this point is $(2,\pi)$.

	\item The distance from the point with rectangular coordinates $(-1,1)$ to the origin is $\sqrt{2}$, and the angle the line through the origin and this point makes with the positive $x$-axis is $\frac{3\pi}{4}$, so the polar coordinate representation of this point is $\left(\sqrt{2},\frac{3\pi}{4}\right)$.


        \end{enumerate}

	\item For each of the following points whose coordinates are given in polar form, determine the rectangular coordinates of the point.
       \begin{enumerate}[i.]
          \item The rectangular coordinates are $x = 5 \cos\left(\frac{\pi}{4}\right) = \frac{5\sqrt{2}}{2}$ and $y = 5 \sin\left(\frac{\pi}{4}\right) = \frac{5\sqrt{2}}{2}$.
         \item  The rectangular coordinates are $x = 2 \cos\left(\frac{5\pi}{6}\right) = -\sqrt{3}$ and $y = 2 \sin\left(\frac{5\pi}{6}\right) = 1$.
		\item The rectangular coordinates are $x = \sqrt{3} \cos\left(\frac{5\pi}{3}\right) = \frac{\sqrt{3}}{2}$ and $y = \sqrt{3} \sin\left(\frac{5\pi}{3}\right) = -\frac{3}{2}$. 
       \end{enumerate}

    \ea


\end{activitySolution} 

\afterpa 