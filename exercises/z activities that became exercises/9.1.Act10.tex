
\begin{activity} \label{A:9.1.10}
The Ideal Gas Law $PV = RT$ relates the pressure ($P$, in pascals), temperature ($T$, in Kelvin), and volume ($V$, in cubic meters) of 1 mole of a gas ($R =  8.314 \ \frac{\text{J}}{\text{mol} \ \text{K}}$ is the universal gas constant), and describes the behavior of gases that do not liquefy easily, such as oxygen and hydrogen. We can solve the ideal gas law for the volume and treat the volume as a function of the pressure and temperature:
\[V(P,T) = \frac{8.314T}{P}.\]
    \ba
    \item Explain in detail what the trace of $V$ with $P=1000$ tells us.
    \item Explain in detail what the trace of $V$ with $T=5$ tells us.
    \item Explain in detail what the level curve $V = 0.5$ tells us.
    \ea
\end{activity}
\begin{smallhint}

\end{smallhint}
\begin{bighint}

\end{bighint}
\begin{activitySolution}


\end{activitySolution}


\aftera 