\begin{exercises} 

\item Evaluate each of the following double or integrated integrals exactly.
\ba
	\item $\ds \int_1^3 \left( \int_2^5 xy \, dy \right) \, dx$
	\item $\ds \int_0^{\pi/4} \left( \int_0^{\pi/3} \sin(x) \cos(y) \, dx \right) \, dy$
	\item $\ds \int_0^1 \left( \int_0^1 e^{-2x - 3y} \, dy \right) \, dx$
	\item $\ds \iint_R \sqrt{2x + 5y} \, dA$, where $R = [0,2]\times[0,3]$.
\ea

\begin{exerciseSolution}
\ba
	\item We integrate first with respect to $y$, then with respect to $x$:
\begin{align*}
\int_1^3 \left( \int_2^5 xy \, dy \right) \, dx &= \int_1^3 \left( x\frac{y^2}{2}\right)\biggm|_{y=2}^{5} \, dx \\
	&= \int_1^3  \frac{21}{2}x \, dx  \\
	&= \frac{21}{4}x^2 \biggm|_{x=1}^3 \\
	&= \frac{21}{4}(8) \\
	&= 42.
\end{align*}

	\item We integrate first with respect to $x$, then with respect to $y$:
\begin{align*}
\int_0^{\pi/4} \left( \int_0^{\pi/3} \sin(x) \cos(y) \, dx \right) \, dy &= \int_0^{\pi/4} \left( -\cos(x) \cos(y) \biggm|_0^{x=\pi/3}  \right) \, dy  \\
	&= \int_0^{\pi/4} \left( -\frac{1}{2}-(-1)\right) \cos(y) \, dy  \\
	&= \frac{1}{2} \sin(y) \biggm|_{y=0}^{\pi/4} \\
	&= \frac{\sqrt{2}}{4}.
\end{align*}

	\item We integrate first with respect to $y$, then with respect to $x$:
\begin{align*}
\int_0^1 \left( \int_0^1 e^{-2x - 3y} \, dy \right) \, dx &= \int_0^1 \left( -\frac{1}{3}e^{-2x - 3y}\biggm|_{y=0}^1 \right) \, dx  \\
	&= -\frac{1}{3} \int_0^{1} \left( e^{-2x-3} - e^{-2x} \right) \, dy  \\
	&= \frac{1}{6} \left(e^{-2x-3} - e^{-2x}\right) \biggm|_{x=0}^{1} \\
	&= \frac{\sqrt{1}}{6} \left(e^{-5}-e^{-2} - e^{-3} + 1\right).
\end{align*}

	\item We integrate first with respect to $y$, then with respect to $x$:
\begin{align*}
\iint_R \sqrt{2x + 5y} \, dA &= \int_0^2 \left( \int_0^3 \sqrt{2x+5y} \, dy \right) \, dx \\
	&= \int_0^2 \left( \frac{2}{15}(2x+5y)^{3/2}\biggm|_{y=0}^3 \right) \, dx \\
	&= \frac{2}{15}\int_0^2 \left( (2x+15)^{3/2} - (2x)^{3/2} \right) \, dx \\
	&= \left(\frac{2}{15}\right)\left(\frac{1}{5}\right) \left( (2x+15)^{5/2} - (2x)^{5/2} \right)\biggm|_{x=0}^2  \\
	&= \frac{2}{75} \left(19^{5/2} - 4^{5/2} - 15^{5/2}\right).
\end{align*}
\ea
\end{exerciseSolution}

\item The temperature at any point on a metal plate in the $xy$ plane is given by $T(x,y) = 100-4x^2 - y^2$, where $x$ and $y$ are measured in inches and $T$ in degrees Celsius.  Consider the portion of the plate that lies on the rectangular region $R = [1,5] \times [3,6]$.

\ba
	\item Write an iterated integral whose value represents the volume under the surface $T$ over the rectangle $R$. 
	\item Evaluate the iterated integral you determined in (a).
	\item Find the area of the rectangle, $R$.
	\item Determine the exact average temperature, $T_{\mbox{\tiny{AVG}(R)}}$, over the region $R$.
\ea

\begin{exerciseSolution}
\ba
	\item An iterated integral whose value represents the volume under the surface $T$ over the rectangle $R$ is
\[\int_1^5 \left( \int_3^6 100 - 4x^2-y^2 \, dy \right) \, dx.\]

	\item Integrating with respect to $y$ then $x$ gives us
\begin{align*}
\int_1^5 \left( \int_3^6 100 - 4x^2-y^2 \, dy \right) \, dx &= \int_1^5 \left( (100-4x^2)y - \frac{y^3}{3} \right)\biggm|_3^6  \, dx \\
	&= \int_1^5 \left( 300-12x^2 - \frac{189}{3} \right) \, dx \\
	&= \left( 300x-4x^3 - \frac{189}{3}x \right)\biggm|_1^5  \\
	&= 1200 - 496 - \frac{756}{3} \\
	&= 452.
\end{align*}

	\item The area of the rectangle, $R$, is $(4)(3) = 12$. 
	
	\item The exact average temperature, $T_{\mbox{\tiny{AVG}(R)}}$, over the region $R$ is
\[T_{\mbox{\tiny{AVG}(R)}} = \frac{1}{\text{Area}(R)} \iint_R T(x,y) \, dA = \frac{452}{12} = \frac{113}{3}.\]

\ea
\end{exerciseSolution}

\item Consider the box with a sloped top that is given by the following description:  the base is the rectangle $R = [1,4] \times [2,5]$, while the top is given by the plane $z = p(x,y) = 30 - x - 2y$.

\ba
	\item Write an iterated integral whose value represents the volume under $p$ over the rectangle $R$. 
	\item Evaluate the iterated integral you determined in (a).
	\item What is the exact average value of $p$ over $R$?
	\item If you wanted to build a rectangular box (with an identical base) that has the same volume as the box with the sloped top described here, how tall would the rectangular box have to be?
\ea

\begin{exerciseSolution}
\ba
	\item An iterated integral whose value represents the volume under $p$ over the rectangle $R$ is
\[\int_1^4 \left( \int_2^5 30-x-2y \, dy \right) \, dx.\] 

	\item Integrating with respect to $y$ then $x$ gives us
\begin{align*}
\int_1^4 \left( \int_2^5 30-x-2y \, dy \right) \, dx &= \int_1^4 \left( (30-x)y - y^2 \right)\biggm|_2^5 \, dx \\
	&= \int_1^4 \left( 90-3x - 21 \right) \, dx \\
	&= \left( 69x-\frac{3}{2}x^2 \right)\biggm|_1^4  \\
	&= 207 - \frac{45}{2}  \\
	&= 184.5.
\end{align*}

	\item The exact average value of $p$ over $R$ is 
\[p_{\mbox{\tiny{AVG}(R)}} = \frac{1}{\text{Area}(R)} \iint_R p(x,y) \, dA = \frac{184.5}{(3)(3)} = 20.5.\]
	\item The average value of $p$ over $R$ tells us the height of a rectangular box (with an identical base) that has the same volume as the box with the sloped top described here, so we need a box of height $20.5$. 
\ea
\end{exerciseSolution}


\end{exercises} 

\afterexercises
