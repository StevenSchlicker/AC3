\begin{exercises} 

\item A triangular plate is bounded by the graphs of the equations $y = 2x$, $y = 4x$, and $y = 4$.  The plate's density at $(x,y)$ is given by  $\delta(x,y) = 4xy^2 + 1$, measured in grams per square centimeter (and $x$ and $y$ are measured in centimeters).  
	
\ba
	\item Set up an iterated integral whose value is the mass of the plate. Include a labeled sketch of the region of integration.  Why did you choose the order of integration you did?  
	\item Determine the mass of the plate.    
	\item Determine the exact center of mass of the plate.  Draw and label the point you find on your sketch from (a).	
	\item  What is the average density of the plate?  Include units on your answer.
\ea

\begin{exerciseSolution}
\ba
	\item We obtain mass by integrating density, so the mass of the plate is
\[\int_{y=0}^{y=4} \int_{x=y/4}^{x=y/2} 4xy^2 + 1 \, dx \, dy.\]
If we instead integrate first with respect to $y$, the fact that the curve on the bottom will change over the interval means that will would need two iterated integrals instead of just one.   
	\item To find the mass $m$ we evaluate the iterated integral:
\begin{align*}
m&=\int_{y=0}^{y=4} \int_{x=y/4}^{x=y/2} 4xy^2 + 1 \, dx \, dy \\
	&= \int_{y=0}^{y=4} \left(2x^2y^2+x\right)\biggm|_{x=y/4}^{x=y/2} \, dy \\
	&= \int_{y=0}^{y=4} \left[\left(\frac{y^4}{2} + \frac{y}{2} \right) - \left(\frac{y^4}{8}+\frac{y}{4} \right)  \right] \, dy \\
	&= \int_{y=0}^{y=4} \left[\frac{3y^4}{8}+\frac{y}{4} \right] \, dy \\
	&= \left[\frac{3}{40}y^5 + \frac{y^2}{8} \right]  \int_{y=0}^{y=4}   \\
	&= \frac{3}{40}4^5 + \frac{4^2}{8}     \\
	&= \frac{394}{5}.
\end{align*}

	\item The center of mass of the plate is $(\overline{x}, \overline{y})$, where
\begin{align*}
\overline{x} &= \frac{1}{m} \int_{y=0}^{y=4} \int_{x=y/4}^{x=y/2} x(4xy^2 + 1) \, dx \, dy \\
	&= \frac{5}{394} \int_{y=0}^{y=4} \left(\frac{4}{3}x^3y^2+\frac{1}{2}x^2\right)\biggm|_{x=y/4}^{x=y/2} \, dy \\
	&= \frac{5}{394} \int_{y=0}^{y=4} \left[\left(\frac{1}{3}\frac{y^5}{2} + \frac{y^2}{8} \right) - \left(\frac{1}{3}\frac{y^5}{16}+\frac{y^2}{32} \right)  \right] \, dy \\
	&= \frac{5}{394} \int_{y=0}^{y=4} \left[\frac{7y^5}{48}+\frac{3y^2}{32} \right] \, dy \\
	&= \frac{5}{394} \left[\frac{7}{288}y^6 + \frac{y^3}{32} \right]  \int_{y=0}^{y=4}   \\
	&= \frac{5}{394} \left[\frac{7}{288}4^6 + \frac{4^3}{32}\right]     \\
	&= \frac{2285}{1773} \\
	&\approx 1.289.
\end{align*}
and
\begin{align*}
\overline{y} &= \frac{1}{m} \int_{y=0}^{y=4} \int_{x=y/4}^{x=y/2} y(4xy^2 + 1) \, dx \, dy \\
	&= \frac{5}{394} \int_{y=0}^{y=4} \left(2x^2y^3+xy\right)\biggm|_{x=y/4}^{x=y/2} \, dy \\
	&= \frac{5}{394} \int_{y=0}^{y=4} \left[\left(\frac{1}{2}y^5 + \frac{1}{2}y^2\right) - \left(\frac{1}{8}y^5+\frac{1}{4}y^2 \right)  \right] \, dy \\
	&= \frac{5}{394} \int_{y=0}^{y=4} \left[\frac{3}{8}y^5+\frac{1}{4}y^2 \right] \, dy \\
	&= \frac{5}{394} \left[\frac{1}{16}y^6 + \frac{1}{12}y^3 \right]  \int_{y=0}^{y=4}   \\
	&= \frac{5}{394} \left[\frac{1}{16}4^6 + \frac{1}{12}4^3 \right]     \\
	&= \frac{1960}{591} \\
	&\approx 3.316.
\end{align*}


	\item First note that the area $A$ of the plate is 
\begin{align*}
A &= \int_{y=0}^{y=4} \int_{x=y/4}^{x=y/2} 1 \, dx \, dy \\
	&= \int_{y=0}^{y=4} x\biggm|_{x=y/4}^{x=y/2}  \, dy \\
	&= \int_{y=0}^{y=4} \frac{y}{4}  \, dy \\
	&= \frac{y^2}{8}\biggm|_{y=0}^{y=4} \\
	&= 2.
\end{align*}
The average density of the plate is then 
\[\frac{1}{A} \int_{y=0}^{y=4} \int_{x=y/4}^{x=y/2} 4xy^2 + 1 \, dx \, dy = \frac{197}{5} \ \frac{\text{g}}{\text{cm}^2}.\]

\ea
\end{exerciseSolution}

\item Let $D$ be a half-disk lamina of radius 3 in quadrants IV and I, centered at the origin as in Activity \ref{A:11.4.3}.  Assume the density at point $(x,y)$ is equal to $x$. 
	\ba 
	    \item Before doing any calculations, what do you expect the $y$-coordinate of the center of mass to be? Why? 
	    \item Set up iterated integral expressions which, if evaluated, will determine the exact center of mass of the lamina. 
	    \item Use appropriate technology to evaluate the integrals to find the center of mass numerically.
	 \ea 
\begin{exerciseSolution}
	\ba
	\item Due to the symmetry of the lamina around the $x$-axis we should expect the $y$-coordinate of the center of mass to be 0. 

	\item We calculated that mass of the lamina to be 18 in Activity \ref{A:11.4.3}. So the $x$-coordinate $\overline{x}$ of the center of mass of the lamina is given by 
\[\overline{x} = \frac{1}{18} \int_{-3}^{3} \int_{0}^{\sqrt{9-y^2}} x^2 \, dx \, dy.\]
Similarly, the $y$-coordinate $\overline{y}$ of the center of mass of the lamina is given by  
\[\overline{y} = \frac{1}{18} \int_{-3}^{3} \int_{0}^{\sqrt{9-y^2}} xy \, dx \, dy.\]

	\item Wolfram$|$Alpha gives $\overline{x} = \frac{9}{16} \pi$ and $\overline{y} = 0$, as expected.   
	\ea
\end{exerciseSolution}

\item Let $x$ denote the time (in minutes) that a person spends waiting in a checkout line at a grocery store and $y$ the time (in minutes) that it takes to check out.  Suppose the joint probability density for $x$ and $y$ is
\[f(x,y) = \frac{1}{8} e^{-x/4-y/2}.\]

	\ba
		\item What is the exact probability that a person spends between 0 to 5 minutes waiting in line, and then 0 to 5 minutes waiting to check out?
		\item Set up, but do not evaluate, an iterated integral whose value determines the exact probability that a person spends at most 10 minutes total both waiting in line and checking out at this grocery store.
		\item Set up, but do not evaluate, an iterated integral expression whose value determines the exact probability that a person spends at least 10 minutes total both waiting in line and checking out, but not more than 20 minutes.
	\ea

\begin{exerciseSolution}
	\ba
	\item We want to have $0 \leq x \leq 5$ and $0 \leq y \leq 5$. So the probability that a person spends between 0 to 5 minutes waiting in line, and then 0 to 5 minutes waiting to check out is
	\begin{align*}
	\int_0^{5} \int_0^{5} f(x,y) \, dy \, dx &= \int_0^{5} \int_0^{5} \frac{1}{8} e^{-x/4-y/2} \, dy \, dx \\
		&= \int_0^{5} -\frac{1}{4} e^{-x/4-y/2}\biggm|_0^{5}  \, dx \\
		&= \int_0^{5} -\frac{1}{4} \left(e^{-x/4-5/2} - e^{-x/4}\right)  \, dx \\
		&= \left(e^{-x/4-5/2} - e^{-x/4}\right)\biggm|_{0}^{5} \\
		&= \left(e^{-15/4} - e^{-5/4}\right) - \left(e^{-5/2} - 1\right) \\
		&\approx 0.655.
	\end{align*}
So the probability is approximately 65.5\%. 

	\item We need to have $x+y = 10$ with $x$ between 0 and $10$, so an iterated integral that represents the probability that a person spends no more than 10 minutes waiting and then checking out at this grocery store is
\[\int_0^{10} \int_0^{10-x} f(x,y) \, dy \, dx.\]

	\item We need to have $10 \leq x+y \leq 20$ with $x$ between 0 and $20$, so an iterated integral that represents the probability that a person spends at least 10 minutes total both waiting in line and checking out, but not more than 20 minutes is 
\[\int_{10}^{20} \int_{10-x}^{20-x} f(x,y) \, dy \, dx.\]

	\ea
\end{exerciseSolution}

\end{exercises}

\afterexercises
