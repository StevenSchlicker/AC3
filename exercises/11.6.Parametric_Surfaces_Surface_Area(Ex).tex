\begin{exercises} 

\item Consider the ellipsoid given by the equation 
$$\frac{x^2}{16} + \frac{y^2}{25} + \frac{z^2}{9} = 1.$$
In Activity~\ref{A:11.6.10}, we found that a parameterization of the sphere $S$ of radius $R$ centered at the origin is 
\[x(r,s) = R\cos(s) \cos(t), \ y(s,t) = R \cos(s) \sin(t), \ \text{ and } \ z(s,t) = R\sin(s)\]
for $-\frac{\pi}{2} \leq s \leq \frac{\pi}{2}$ and $0 \leq t \leq 2\pi$.
	
\ba
	\item Let $(x,y,z)$ be a point on the ellipsoid and let $X = \frac{x}{4}$, $Y = \frac{y}{5}$, and $Z = \frac{z}{3}$. Show that $(X,Y,Z)$ lies on the sphere $S$. Hence, find a parameterization of $S$ in terms of $X$, $Y$, and $Z$ as functions of $s$ and $t$. 
	\item Use the result of part (a) to find a parameterization of the ellipse in terms of $x$, $y$, and $z$ as functions of $s$ and $t$. Check your parametrization by substituting $x$, $y$, and $z$ into the equation of the ellipsoid. Then check your work by plotting the surface defined by your parameterization with
    appropriate technology\footnote{e.g.,
    \url{http://web.monroecc.edu/manila/webfiles/calcNSF/JavaCode/CalcPlot3D.htm}
    or
    \url{http://www.flashandmath.com/mathlets/multicalc/paramrec/surf_graph_rectan.html}}.
\ea

\begin{exerciseSolution}
\ba
	\item If $(x,y,z)$ lies on the ellipse and $X = \frac{x}{4}$, $Y = \frac{y}{5}$, and $Z = \frac{z}{3}$, then 
\begin{align*}
X^2+Y^2+Z^2 &= \left(\frac{x}{4}\right)^2 + \left(\frac{y}{5}\right)^2 + \left(\frac{z}{3}\right)^2 \\
	&= \frac{x^2}{16} + \frac{y^2}{25} + \frac{z^2}{9} \\
	&= 1.
\end{align*}
So $(X,Y,Z)$ lies on the sphere $S$ and $S$ has parameterization 
\[X(r,s) = \cos(s) \cos(t), \ Y(s,t) = \cos(s) \sin(t), \ \text{ and } \ Z(s,t) = \sin(s)\]
for $-\frac{\pi}{2} \leq s \leq \frac{\pi}{2}$ and $0 \leq t \leq 2\pi$.
	\item Since $x = 4X$, $y = 5Y$, and $z = 3Z$ we have 
\[x(r,s) = 4\cos(s) \cos(t), \ y(s,t) = 5\cos(s) \sin(t), \ \text{ and } \ z(s,t) = 3\sin(s)\]
for $-\frac{\pi}{2} \leq s \leq \frac{\pi}{2}$ and $0 \leq t \leq 2\pi$. Note that 
\begin{align*}
\frac{x^2}{16} + \frac{y^2}{25} + \frac{z^2}{9} &= \cos^2(s)\cos^2(t) + \cos^2(s) \sin^2(t)  + \sin^2(s) \\
	&= \cos^2(s)[\cos^2(t) + \sin^2(t)] + \sin^2(s) \\
	&= \cos^2(s) + \sin^2(s) \\
	&= 1.
\end{align*}
So $x(r,s) = 4\cos(s) \cos(t)$, $y(s,t) = 5\cos(s) \sin(t)$, and $z(s,t) = 3sin(s)$ is a parametrization of the ellipse. 
\ea
\end{exerciseSolution}

  \item In this exercise, we explore how to use a parametrization and iterated integral to determine the surface area of a sphere.
  \ba
  	\item Set up an iterated integral whose value is the portion of the surface area of a sphere of radius $R$ that lies in the first octant (see the parameterization you developed in Activity \ref{A:11.6.10}). 
	\item Then, evaluate the integral to calculate the surface area of this portion of the sphere.
	\item By what constant must you multiply the value determined in (b) in order to find the total surface area of the entire sphere.  	
\item Finally, compare your result to the standard formula for the surface area of sphere. 
  \ea

\begin{exerciseSolution}
\ba
\item From Activity \ref{A:11.6.10} we know that a parameterization of a sphere of radius $R$ centered at the origin is 
\[x(s,t) = R \cos(s) \cos(t), \ y(s,t) = R\sin(s) \cos(t), \ \text{ and } \ z(s,t) = R \sin(t)\]
for $-\frac{\pi}{2} \leq s \leq \frac{\pi}{2}$ and $t$ in $[0, 2 \pi]$. By symmetry, we can just calculate the surface area of the sphere in the first octant, then multiply by 8. So we can assume $0 \leq s, t \leq \frac{\pi}{2}$.
\item With $\vr(s,t) = \langle R \cos(s) \cos(t),  R\sin(s) \cos(t),  R \sin(t) \rangle$ we have
\begin{align*}
\vr_s(s,t) &= \langle -R\sin(s) \cos(t), R\cos(s)\cos(t), 0 \rangle, \\
\vr_t(s,t) &= \langle -R\cos(s) \sin(t), -R\sin(s)\sin(t), -R\cos(t) \rangle,
\end{align*}
and so 
\begin{align*}
\lvert \vr_s(s,t) \times \vr_t(s,t) \rvert &= \lvert \langle -R\sin(s) \cos(t), R\cos(s)\cos(t), 0 \rangle \times \langle -R\cos(s) \sin(t), -R\sin(s)\sin(t), -R\cos(t) \rangle \rvert \\
    &= \lvert \langle -R^2\cos(s)\cos^2(t), -R^2\sin(s)\cos^2(t), R^2\cos(t)\sin(t) \rangle \rvert \\
    &= \sqrt{R^4\cos^2(s)\cos^4(t) + R^4\sin^2(s)\cos^4(t) + R^4\cos^2(t)\sin^2(t)} \\
    &= R^2\sqrt{\cos^2(t)} \\
    &= R^2\cos(t).
\end{align*}
Therefore, the surface area of the first octant portion of the sphere is
\begin{align*}
 \int_0^{\pi/2} \int_0^{\pi/2} R^2 \cos(t) \, dt \, ds &= 8R^2 \int_0^{\pi/2} \sin(t) \biggm|_0^{\pi/2} \, ds \\
    &= R^2 \int_0^{\pi/2} 1 \, ds \\
    &= \frac{1}{2}\pi R^2.
\end{align*}
\item We multiply the previous result by 8 to reproduce the surface area calculation in each of the eight octants. So the surface area of a sphere of radius $R$ is $4 \pi R^2$. 
\item  The result of part (c) is the standard formula for the surface area of a sphere of radius $R$. 
\ea
\end{exerciseSolution}

\item Consider the plane generated by $z = f(x,y) = 24 - 2x - 3y$ over the region $D = [0,2]\times[0,3]$.
	
\ba
	\item Sketch a picture of the overall solid generated by the plane over the given domain.
	\item Determine a parameterization $\vr(s,t)$ for the plane over the domain $D$.
	\item Use Equation~\ref{E:surface_area} to determine the surface area generated by $f$ over the domain $D$.
	\item Observe that the vector $\vu = \langle 2, 0, -4 \rangle$ points from $(0,0,24)$ to $(2,0,20)$ along one side of the surface generated by the plane $f$ over $D$.  Find the vector $\vv$ such that $\vu$ and $\vv$ together span the parallelogram that represents the surface defined by $f$ over $D$, and hence compute $| \vu \times \vv |$.  What do you observe about the value you find?
\ea

\begin{exerciseSolution}
\ba
	\item The solid looks like a box with a slanted top. 
	\item If we let $x(s,t) = s$ and $y(s,t) = t$, then $z(s,t) = 24-2s-3t$ provides a parameterization of the plane. So we can let $\vr(s,t) = s \vi + t \vj + (24-2s-3t) \vk$. 
	\item In our situation we have $\vr_s(s,t) = \vi - 2 \vk$ and $\vr_t(s,t) = \vj - 3 \vk$. So $\vr_s(s,t) \times \vr_t(s,t) = 2 \vi + 3\vj + \vk$. So  the surface area generated by $f$ over the domain $D$ is found by 
\begin{align*}
 \int_0^{3} \int_0^{2} | \langle 2,3,1 \rangle | \, dt \, ds &=  \sqrt{14} \int_0^{3} \int_0^{2} \, dt \, ds \\
    &= 6 \sqrt{14}.
\end{align*}
	\item Take $\vv = \langle 0, 2, -9 \rangle$ to be the vector that points from $(0,0,24)$ to $(0,3,15)$ along one side of the surface generated by the plane $f$ over $D$. Then $\vu$ and $\vv$ together span the parallelogram that represents the surface defined by $f$ over $D$. Here we have 
\[| \vu \times \vv | = | \langle 12, 18, 6 \rangle | = \sqrt{504} = 6 \sqrt{14}.\]
Since $| \vu \times \vv |$ is the area of the parallelogram determined by $\vu$ and $\vv$, we should expect this area to be the same as the surface area of the surface defined by the plane $f$ over $D$. 
\ea
\end{exerciseSolution}

\item A cone with base radius $a$ and height $h$ can be realized as the surface defined by $z = \frac{h}{a} \sqrt{x^2+y^2}$, where $a$ and $h$ are positive.  
 	\ba
	\item Find a parameterization of the cone described by $z = \frac{h}{a} \sqrt{x^2+y^2}$. (Hint: Compare to the parameterization of a cylinder as seen in Activity  \ref{A:11.6.4}.) 
	
	\item Set up an iterated integral to determine the surface area of this cone.
	
	\item Evaluate the iterated integral to find a formula for the lateral surface area of a cone of height $h$ and base $a$. 
	
	\ea

\begin{exerciseSolution}
\ba
\item Let $t$ run along the positive $z$-axis from 0 to $h$. By similar triangles, the cross section of the cone at height $t$ parallel to the $x$-$y$ plane has radius $\frac{h}{a}t$. We can think of the cone as made of circles with these varying radii, so a parameterization of the cone is 
\[\vr(s,t) = \frac{h}{a}t \cos(s) \vi + \frac{h}{a}t \sin(s) \vj + t \vk,\]
for $0 \leq s \leq 2 \pi$ and $0 \leq t \leq a$. 

\item We have 
\[\vr_s(s,t) = -\frac{h}{a}t\sin(s) \vi + \frac{h}{a}t\cos(s) \vj  \ \text{ and } \ \vr_t(s,t) =  \frac{h}{a} \cos(s) \vi + \frac{h}{a} \sin(s) \vj + \vk,\]
and
\[\vr_s(s,t) \times \vr_t(s,t) = -\frac{h}{a}t\cos(s) \vi + \frac{h}{a}t\sin(s) \vj + \left(\frac{a}{h}\right)^2 t \vk.\]
By symmetry, we calculate the area of the cone in the first octant and multiply by 4 to obtain the area of the surface of the cone as
\[4\int_0^{\pi/2} \int_0^a |\vr_s \times \vr_t| \, dt \, ds = 4\int_0^{\pi/2} \int_0^a \frac{a}{h^2} \sqrt{h^2+a^2} t \, dt \, ds.\]

\item Evaluating the iterated integral yields
\begin{align*}
4\int_0^{\pi/2} \int_0^a \frac{a}{h^2} \sqrt{h^2+a^2} t \, dt \, ds &= 4\frac{a}{h^2} \sqrt{h^2+a^2} \int_0^{\pi/2} \frac{1}{2}t^2 \biggm|_0^h \, ds \\
	&= 2 a \sqrt{h^2+a^2} \int_{0}^{\pi/2}  \, ds \\
	&= \pi  a \sqrt{h^2+a^2}. 
\end{align*}

\ea
\end{exerciseSolution}
\end{exercises}
\afterexercises
