\begin{exercises} 

\item The temperature at any point on a metal plate in the $xy$ plane is given by $T(x,y) = 100-4x^2 - y^2$, where $x$ and $y$ are measured in inches and $T$ in degrees Celsius.  Consider the portion of the plate that lies on the rectangular region $R = [1,5] \times [3,6]$.

\ba
	\item Estimate the value of $\iint_R T(x,y) \, dA$ by using a double Riemann sum with two subintervals in each direction and choosing $(x_i^*, y_j^*)$ to be the point that lies in the upper right corner of each subrectangle.  
	\item Determine the area of the rectangle $R$.
	\item Estimate the average temperature, $T_{\mbox{\tiny{AVG}(R)}}$, over the region $R$.
	\item Do you think your estimate in (c) is an over- or under-estimate of the true temperature?  Why?
\ea

\begin{exerciseSolution}
\ba
	\item The partition points in the $x$ direction are $1$, $3$, and $5$ and the partition points in the $y$ direction are $3$, $4.5$, and $6$. Using the points that lie in the upper right corner of each subrectangle we have
\[\iint_R T(x,y) \, dA \approx T(3,4.5)(3) + T(5,4.5)(3) + T(3,6)(3) + T(5,6)(3) = 46.5.\]

	\item The area of the rectangle $R$ is $(4)(3) = 12$. 

	\item Using the result of part (a) we approximate the average temperature over $R$ as 
\[T_{\mbox{\tiny{AVG}(R)}} = \frac{1}{\text{Area}(R)} \iint_R T(x,y) \approx \frac{46.5}{12} = 3.875.\]

	\item The temperature function is decreasing in both the $x$ and $y$ directions, so the value at the upper right corner of each subrectangle is the smallest temperature on that subrectangle. So our approximation is an underestimate. 
\ea
\end{exerciseSolution}

\item The wind chill, as frequently   reported, is a measure of how cold it feels outside when the wind is blowing.  In Table \ref{T:11.1.Ex.wind.chill}, the wind chill $w=w(v,T)$, measured in degrees Fahrenheit, is a function of the wind speed $v$, measured in miles per hour, and the ambient air temperature $T$, also measured in degrees Fahrenheit. Approximate the average wind chill on the rectangle $[5,35] \times [-20,20]$ using 3 subintervals in the $v$ direction, 4 subintervals in the $T$ direction, and the point in the lower left corner in each subrectangle. 
\begin{table}[ht] 
  \begin{center}
    \begin{tabular}{|c||c|c|c|c|c|c|c|c|c|}
      \hline
      $v \backslash T$  
         &-20 &-15 &-10 &-5  &0   &5   &10  &15  &20  \\
      \hhline{|=||=|=|=|=|=|=|=|=|=|}
      5  &-34 &-28 &-22 &-16 &-11 &-5 &1 &7 &13  \\
      \hline
      10 &-41 &-35 &-28 &-22 &-16 &-10 &-4 &3 &9   \\
      \hline
      15  &-45 &-39 &-32 &-26 &-19 &-13 &-7 &0 &6  \\
      \hline
      20 &-48 &-42 &-35 &-29 &-22 &-15 &-9 &-2 &4  \\
      \hline
      25 &-51 &-44 &-37 &-31 &-24 &-17 &-11 &-4 &3 \\
      \hline
      30  &-53 &-46 &-39 &-33 &-26 &-19 &-12 &-5 &1 \\
      \hline
      35  &-55 &-48 &-41 &-34 &-27 &-21 &-14 &-7 &0 \\
      \hline
    \end{tabular}
    \caption{Wind chill as a function of wind speed and temperature.}
    \label{T:11.1.Ex.wind.chill}
  \end{center}
\end{table}

\begin{exerciseSolution}
The length of each subinterval in the $v$ direction is $\frac{35-5}{3} = 10$, so our partition points in the $v$ direction are $5$, $15$, $25$, and $35$. The length of each subinterval in the $T$ direction is $\frac{20-(-20)}{4} = 10$, so our partition points in the $T$ direction are $-20$, $-10$, $0$, $10$, and $20$. So
\begin{align*}
\iint w(v,t) \, dA &\approx w(5,-20)(100) + w(15,-20)(100) + w(25,-20)(100) \\
	&\qquad + w(5,-10)(100) + w(15,-10)(100) + w(25,-10)(100) \\
	&\qquad + w(5,0)(100) + w(15,0)(100) + w(25,0)(100)  \\
	&\qquad + w(5,10)(100) + w(15,10)(100) + w(25,10)(100) \\
	&= -29200.
\end{align*}
So the average wind chill on $R$ is approximately
\[ \frac{1}{\text{Area}(R)} \iint_R w(v,T) \approx -\frac{29200}{(30)(40)} = -\frac{73}{3} \approx -24.33.\]
	
\end{exerciseSolution}

\item Consider the box with a sloped top that is given by the following description:  the base is the rectangle $R = [0,4] \times [0,3]$, while the top is given by the plane $z = p(x,y) = 20 - 2x - 3y$.

\ba
	\item Estimate the value of $\iint_R p(x,y) \, dA$ by using a double Riemann sum with four subintervals in the $x$ direction and three subintervals in the $y$ direction, and choosing $(x_i^*, y_j^*)$ to be the point that is the midpoint of each subrectangle.  
	\item What important quantity does your double Riemann sum in (a) estimate?
	\item Suppose it can be determined that $\iint_R p(x,y) \, dA = 138$.  What is the exact average value of $p$ over $R$?
	\item If you wanted to build a rectangular box (with the same base) that has the same volume as the box with the sloped top described here, how tall would the rectangular box have to be?
\ea

\begin{exerciseSolution}
\ba
	\item Our partition points in the $x$ direction are $0$, $1$, $2$, $3$, and $4$ while the partition points in the $y$ direction are $0$, $1$, $2$, and $3$. So 
\begin{align*}
\iint_R p(x,y) \, dA \approx p(0.5,0.5)(1) &+ p(1.5, 0.5)(1) + p(2.5,0.5)(1) + p(3.5,0.5)(1) \\
	&\qquad + p(0.5,1.5)(1) + p(1.5, 1.5)(1) + p(2.5,1.5)(1) + p(3.5,1.5)(1) \\
	&\qquad + p(0.5,2.5)(1) + p(1.5, 2.5)(1) + p(2.5,2.5)(1) + p(3.5,2.5)(1) \\
	&=  138.
\end{align*}

	\item Since $p(x,y) \geq 0$ on $R$, the geometric quantity our double Riemann sum approximates is the volume of the surface with the plane defined by $p$ as the top and with base $R$. 
 
	\item The area of $R$ is 12, so the exact average value of $p$ over $R$ is 
	\[ \frac{1}{\text{Area}(R)} \iint_R p(x,y) \approx \frac{138}{12} = 11.5.\]
	
	\item If $h$ is the height of the box, then we would need to have $12h=138$, or $h = 11.5$. This is the average value of $p$ over $R$. 
\ea
\end{exerciseSolution}



\end{exercises}
\afterexercises
