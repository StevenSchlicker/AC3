\begin{exercises} 
\item Suppose that $V(t) = 24 \cdot 1.07^t + 6 \sin(t)$ represents the value of a person's investment portfolio in thousands of dollars in year $t$, where $t = 0$ corresponds to January 1, 2010.
\ba
	\item At what instantaneous rate is the portfolio's value changing on January 1, 2012?  Include units on your answer.
	\item Determine the value of $V''(2)$.  What are the units on this quantity and what does it tell you about how the portfolio's value is changing?
	\item On the interval $0 \le t \le 20$, graph the function $V(t) = 24 \cdot 1.07^t + 6 \sin(t)$ and describe its behavior in the context of the problem.  Then, compare the graphs of the functions $A(t) = 24 \cdot 1.07^t$ and $V(t) = 24 \cdot 1.07^t + 6 \sin(t)$, as well as the graphs of their derivatives $A'(t)$ and $V'(t)$.  What is the impact of the term $6 \sin(t)$ on the behavior of the function $V(t)$?
\ea
\item Let $f(x) = 3\cos(x) - 2\sin(x) + 6$.
	\ba
		\item Determine the exact slope of the tangent line to $y = f(x)$ at the point where $a = \frac{\pi}{4}$.
		\item Determine the tangent line approximation to $y = f(x)$ at the point where $a = \pi$.
		\item At the point where $a = \frac{\pi}{2}$, is $f$ increasing, decreasing, or neither?
		\item At the point where $a = \frac{3\pi}{2}$, does the tangent line to $y = f(x)$ lie above the curve, below the curve, or neither?  How can  you answer this question without even graphing the function or the tangent line?
	\ea
\item In this exercise, we explore how the limit definition of the derivative more formally shows that $\frac{d}{dx}[\sin(x)] = \cos(x)$.   Letting $f(x) = \sin(x)$, note that the limit definition of the derivative tells us that
	$$f'(x) = \lim_{h \to 0} \frac{\sin(x+h) - \sin(x)}{h}.$$
\ba
	\item Recall the trigonometric identity for the sine of a sum of angles $\alpha$ and $\beta$: \\ $\sin(\alpha + \beta) = \sin(\alpha)\cos(\beta) + \cos(\alpha)\sin(\beta)$.  Use this identity and some algebra to show that
	$$f'(x) = \lim_{h \to 0} \frac{\sin(x)(\cos(h)-1) + \cos(x)\sin(h)}{h}.$$
	\item Next, note that as $h$ changes, $x$ remains constant. Explain why it therefore makes sense to say that
	$$f'(x) = \sin(x) \cdot \lim_{h \to 0} \frac{\cos(h) -1 }{h} + \cos(x) \cdot \lim_{h \to 0} \frac{\sin(h)}{h}.$$
	\item Finally, use small values of $h$ to estimate the values of the two limits in (c):
	$$\lim_{h \to 0} \frac{\cos(h) - 1}{h} \ \ \mbox{and} \ \ \lim_{h \to 0} \frac{\sin(h)}{h}.$$
	\item What do your results in (c) thus tell you about $f'(x)$?
\ea 
\end{exercises}
\afterexercises