\begin{exercises} 
  \item Consider the curve $f(x) = 3 \cos(\frac{x^3}{4})$ and the portion of its graph that lies in the first quadrant between the $y$-axis and the first positive value of $x$ for which $f(x) = 0$.  Let  $R$ denote the region bounded by this portion of $f$, the $x$-axis, and the $y$-axis.  Assume that $x$ and $y$ are each measured in feet.
  
  	\ba
		\item Picture the coordinate axes rotated 90 degrees clockwise so that the positive $x$-axis points straight down, and the positive $y$-axis points to the right.  Suppose that $R$ is rotated about the $x$ axis to form a solid of revolution, and we consider this solid as a storage tank.  Suppose that the resulting tank is filled to a depth of 1.5 feet with water weighing 62.4 pounds per cubic foot.  Find the amount of work required to lower the water in the tank until it is 0.5 feet deep, by pumping the water to the top of the tank.
		\item Again picture the coordinate axes rotated 90 degrees clockwise so that the positive $x$-axis points straight down, and the positive $y$-axis points to the right.  Suppose that $R$, together with its reflection across the $x$-axis, forms one end of a storage tank that is 10 feet long.  Suppose that the resulting tank is filled completely with water weighing 62.4 pounds per cubic foot.  Find a formula for a function that tells the amount of work required to lower the water by $h$ feet.
		\item Suppose that the tank described in (b) is completely filled with water.  Find the total force due to hydrostatic pressure exerted by the water on one end of the tank.
	\ea 
  
  \item A cylindrical tank, buried on its side, has radius 3 feet and length 10 feet.  It is filled completely with water whose weight density is 62.4 lbs/ft$^3$, and the top of the tank is two feet underground.

	\ba
		\item Set up, but do not evaluate, an integral expression that represents the amount of work required to empty the top half of the water in the tank to a truck whose tank lies 4.5 feet above ground.
  
		\item With the tank now only half-full, set up, but do not evaluate an integral expression that represents the total force due to hydrostatic pressure against one end of the tank.
	\ea

  
\end{exercises}
\afterexercises
