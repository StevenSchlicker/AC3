\begin{exercises} 
  \item  Congratulations, you just won the lottery!  In one option
    presented to you, you will be paid one million dollars a year for
    the next 25 years.  You can deposit this money in an account that
    will earn 5\% each year.

    \ba
  \item Set up a differential equation that describes the rate of
    change in the amount of money in the account.  Two factors cause
    the amount to grow---first, you are depositing one millon dollars
    per year and second, you are earning 5\% interest.

  \item If there is no amount of money in the account when you open
    it, how much money will you have in the account after 25 years?

  \item The second option presented to you is to take a lump sum of 10
    million dollars, which you will deposit into a similar account.  How
    much money will you have in that account after 25 years?

  \item Do you prefer the first or second option?  Explain your thinking. 

  \item At what time does the amount of money in the account under the
    first option overtake the amount of money in the account under the
    second option?
    \ea

  \item When a skydiver jumps from a plane, gravity causes 
    her downward velocity to increase at the rate of $g\approx 9.8$
    meters per second squared.  At the same time, wind resistance
    causes her velocity to decrease at a rate proportional to the
    velocity.  

    \ba
    \item Using $k$ to represent the constant of proportionality,
      write a differential equation that describes the rate of change
      of the skydiver's velocity.
    \item Find any equilibrium solutions and decide whether they are
        stable or unstable.  Your result should depend on $k$.
      \item Suppose that the initial velocity is zero.  Find the
        velocity $v(t)$.
      \item A typical terminal velocity for a skydiver falling face
        down is 54 meters per second.  What is the value of $k$ for
        this skydiver?
      \item How long does it take to reach 50\% of the terminal
        velocity? 
      \ea

    \item During the first few years of life, the rate at which a baby
      gains rate is proportional to the reciprocal of its weight.

      \ba
      \item Express this fact as a differential equation.

      \item Suppose that a baby weighs 8 pounds at birth and 9 pounds
        one month later.  How much will he weigh at one year?
      \item Do you think this is a realistic model for a long time?
        \ea

\item  Suppose that you have a water tank that holds 100 gallons of water.
  A briny solution, which contains 20 grams of salt per gallon, enters
  the tank at the rate of 3 gallons per minute.

  At the same time, the solution is well mixed, and water is pumped
  out of the tank at the rate of 3 gallons per minute.

\ba
\item Since 3 gallons enters the tank every minute and 3 gallons
  leaves every minute, what can you conclude about the volume of water
  in the tank.

\item How many grams of salt enters the tank every minute?

\item Suppose that $S(t)$ denotes the number of grams of salt in the
  tank in minute $t$.  How many grams are there in each gallon in
  minute $t$?

\item Since water leaves the tank at 3 gallons per minute, how many
  grams of salt leave the tank each minute? 

\item Write a differential equation that expresses the total rate of
  change of $S$.

\item Identify any equilibrium solutions and determine whether they
  are stable or unstable.

\item Suppose that there is initially no salt in the tank.  Find the
  amount of salt $S(t)$ in minute $t$.

\item What happens to $S(t)$ after a very long time?  Explain how you
  could have predicted this only knowing how much salt there is in
  each gallon of the
  briny solution that enters the tank.
\ea

\end{exercises}
\afterexercises


