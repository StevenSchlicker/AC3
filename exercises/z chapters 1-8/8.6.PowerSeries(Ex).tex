\begin{exercises}

\item We can use power series to approximate definite integrals to which known techniques of integration do not apply. We will illustrate this in this exercise with the definite integral $\int_0^1 \sin(x^2) \ ds$.
    \ba
    \item Use the Taylor series for $\sin(x)$ to find the Taylor series for $\sin(x^2)$. What is the interval of convergence for the Taylor series for $\sin(x^2)$? Explain.

    \item Integrate the Taylor series for $\sin(x^2)$ term by term to obtain a power series expansion for $\int \sin(x^2) \ dx$.

    \item Use the result from part (b) to explain how to evaluate $\int_0^1 \sin(x^2) \ dx$. Determine the number of terms you will need to approximate $\int_0^1 \sin(x^2) \ dx$ to 3 decimal places.

    \ea

\item \label{ex:8.5.2} There is an important connection between power series and Taylor series. Suppose $f$ is defined by a power series centered at 0 so that
\[f(x) = \sum_{k=0}^{\infty} a_kx^k.\]
    \ba
    \item Determine the first 4 derivatives of $f$ evaluated at 0 in terms of the coefficients $a_k$.

    \item Show that $f^{(n)}(0) = n!a_n$ for each positive integer $n$.

\begin{exerciseSolution}
Then
\begin{align*}
f'(x) &= \sum_{k=1}^{\infty} ka_kx^{k-1} \\
f''(x) &= \sum_{k=2}^{\infty} k(k-1)a_kx^{k-2} \\
f^{(3)}(x) &= \sum_{k=3}^{\infty} k(k-1)(k-2)a_kx^{k-3} \\
\vdots & \ \qquad \vdots \\
f^{(n)}(x) &= \sum_{k=n}^{\infty} k(k-1)(k-2) \cdots (k-n+1) a_kx^{k-n} \\
\vdots & \ \qquad \vdots \\
\end{align*}
So
\begin{align*}
f(0) &= a_0 \\
f'(0) &= a_1 \\
f''(0) &= 2!a_2 \\
f^{(3)}(0) &= 3!a_3 \\
\vdots & \ \qquad \vdots \\
f^{(k)}(0) &= k!a_k \\
\vdots & \ \qquad \vdots \\
\end{align*}
and
\[a_k = \frac{f^{(k)}(0)}{k!}\]
for each $k \geq 0$. But these are just the coefficients of the Taylor series expansion of $f$, which leads us to the following observation.

\end{exerciseSolution}

    \item Explain how the result of (b) tells us the following:

    \ea

\vspace*{5pt}
\nin \framebox{\hspace*{3 pt}
\parbox{6.25 in}{
On its interval of convergence, a power series is the Taylor series of its sum.
} \hspace*{3 pt}}
\vspace*{1pt}

\item In this exercise we will begin with a strange power series and then find its sum. The Fibonacci sequence $\{f_n\}$ is a famous sequence whose first few terms are 
    \[f_0 = 0, f_1 = 1, f_2 = 1, f_3 = 2, f_4 = 3, f_5 = 5, f_6 = 8, f_7 = 13, \cdots,\]
    where each term in the sequence after the first two is the sum of the preceding two terms. That is, $f_0 = 0$, $f_1 = 1$ and for $n \geq 2$ we have 
    \[f_n = f_{n-1} + f_{n-2}.\]
    Now consider the power series 
    \[F(x) = \sum_{k=0}^{\infty} f_kx^k.\]
    We will determine the sum of this power series in this exercise.
    \ba
    \item Explain why each of the following is true.
        \begin{itemize}
        \item[(i)] $\ds xF(x) = \sum_{k=1}^{\infty} f_{k-1}x^k$
        \item[(ii)] $\ds x^2F(x) = \sum_{k=2}^{\infty} f_{k-2}x^k$
        \end{itemize}
    
    \item Show that  
    \[F(x) - xF(x) - x^2F(x) = x.\]
    
    \item Now use the equation 
    \[F(x) - xF(x) - x^2F(x) = x\]
    to find a simple form for $F(x)$ that doesn't involve a sum. 
    
    \item Use a computer algebra system or some other method to calculate the first 8 derivatives of $\frac{x}{1-x-x^2}$ evaluated at 0. Why shouldn't the results surprise you?
    
    \ea
    
\item Airy's equation\footnote{The general differential equations of the form $y'' \pm k^2xy = 0$ is called Airy's equation. These equations arise in many problems, such as the study of diffraction of light, diffraction of radio waves around an object, aerodynamics, and the buckling of a uniform column under its own weight.}
\begin{equation} \label{eq:PA8.6_Airy_equation}
y'' + xy = 0,
\end{equation}
can be used to model an undamped vibrating spring with spring constant $x$ (note that $y$ is an unknown function of $x$). So the solution to this differential equation will tell us the behavior of a spring-mass system as the spring ages (like an automobile shock absorber). Assume that a solution $y=f(x)$ has a Taylor series that can be written in the form
\[y = \sum_{k=0}^{\infty} a_kx^k,\]
where the coefficients are undetermined. Our job is to find the coefficients.
    \ba
    \item Differentiate the series for $y$ term by term to find the series for $y'$. Then repeat to find the series for $y''$.

\begin{exerciseSolution}
Now
\begin{align*}
y' &= \sum_{k=1}^{\infty} ka_kx^{k-1} \\
y'' &= \sum_{k=2}^{\infty} (k-1)na_kx^{k-2}.
\end{align*}
\end{exerciseSolution}

    \item Substitute your results from part (a) into the Airy equation and show that we can write Equation (\ref{eq:PA8.6_Airy_equation}) in the form
\begin{equation} \label{eq:PA8.6_Airy_1}
\sum_{k=2}^{\infty} (k-1)ka_kx^{k-2} + \sum_{k=0}^{\infty} a_kx^{k+1} = 0.
\end{equation}

\begin{exerciseSolution}

We substitute our series from (a) into the Airy equation to obtain the equation
\[\sum_{k=2}^{\infty} (k-1)ka_kx^{k-2} + x\sum_{k=0}^{\infty} a_kx^{k} = 0.\]
Distributing the $x$ in the second term on the left yields the equation in (\ref{eq:PA8.6_Airy_1}).

\end{exerciseSolution}

    \item At this point, it would be convenient if we could combine the series on the left in (\ref{eq:PA8.6_Airy_1}), but one written with terms of the form $x^{k-2}$ and the other with terms in the form $x^{k+1}$. Explain why
\begin{equation} \label{eq:PA8.6_Airy_sum_1}
\sum_{k=2}^{\infty} (k-1)ka_kx^{k-2} = \sum_{k=0}^{\infty} (k+1)(k+2)a_{k+2}x^{k}.
\end{equation}

\begin{exerciseSolution}

Note that
\begin{align*}
\sum_{k=2}^{\infty} (k-1)ka_kx_{k-2} &= (1)(2)a_2 + (2)(3)a_3x + (3)(4)a_4x^2 + (4)(5)a_5x^3 + \cdots + (k-1)(k)a_{k+2}x^{k-2} + \cdots  \\
    &= (0+1)(0+2)x^0 + (1+1)(1+2)x^1 + (2+1)(2+2)x^2 + \cdots + (k+1)(k+2)x^k + \cdots.
\end{align*}
In other words, we can re-index this series by increasing every $k$ by 2, or replacing $k-2$ with $k$, $k-1$ with $k+1$ and $k$ with $k+2$. This gives us equation (\ref{eq:PA8.6_Airy_sum_1}).

\end{exerciseSolution}

    \item Now show that
\begin{equation} \label{eq:PA8.6_Airy_sum_2}
\sum_{k=0}^{\infty} a_kx^{k+1} = \sum_{k=1}^{\infty} a_{k-1}x^k.
\end{equation}

\begin{exerciseSolution}

As we did in the previous part,
\begin{align*}
\sum_{k=0}^{\infty} a_kx_{k+1} &= a_0x + a_1x^2 + a_2x^3 +  \cdots + a_kx^{k+1} + \cdots \notag \\
    &= a_0x + a_1x^2 + a_2x^3 +  \cdots + a_{k-1}x^{k} + \cdots \notag \\
    &= \sum_{k=1}^{\infty} a_{k-1}x^k.
\end{align*}

\end{exerciseSolution}

    \item We can now substitute (\ref{eq:PA8.6_Airy_sum_1}) and (\ref{eq:PA8.6_Airy_sum_2}) into (\ref{eq:PA8.6_Airy_1}) to obtain
\begin{equation} \label{eq:PA8.6_Airy_2}
\sum_{n=0}^{\infty} (n+1)(n+2)a_{n+2}x^{n} + \sum_{n=1}^{\infty} a_{n-1}x^{n} = 0.
\end{equation}
Combine the like powers of $x$ in the two series to show that our solution must satisfy
\begin{equation} \label{eq:PA8.6_Airy_sum_3}
2a_2 + \sum_{k=1}^{\infty} \left[(k+1)(k+2)a_{k+2}+a_{k-1} \right] x^{k} = 0.
\end{equation}

\begin{exerciseSolution}

Notice that we have like powers of $x$ in our two series, so we can combine them and obtain
\begin{align*}
0 &= \sum_{k=0}^{\infty} (k+1)(k+2)a_{k+2}x^{k} + \sum_{k=1}^{\infty} a_{k-1}x^{k} \\
    &= \left[(1)(2)a_2 + \sum_{k=1}^{\infty} (k+1)(k+2)a_{k+2}x^{k} \right] + \sum_{k=1}^{\infty} a_{k-1}x^{k} \\
    &= 2a_2 + \sum_{k=1}^{\infty} \left[(k+1)(k+2)a_{k+2}+a_{k-1} \right] x^{k}.
\end{align*}

\end{exerciseSolution}

    \item Use equation (\ref{eq:PA8.6_Airy_sum_3}) to show the following:
    \begin{itemize}
    \item[(i)] $a_{3k+2} = 0$ for every positive integer $k$,
    \item[(ii)] $a_{3k} = \frac{1}{(2)(3)(5)(6) \cdots (3k-1)(3k)} a_0 \text{ for } k \geq 1$,
    \item[(iii)] $a_{3k+1} = \frac{1}{(3)(4)(6)(7) \cdots (3k)(3k+1)} a_1 \text{ for } k \geq 1$.
    \end{itemize}

\begin{exerciseSolution}

Equation (\ref{eq:PA8.6_Airy_sum_3}) implies that
\[a_2 = 0 \ \text{ and } \ (k+1)(k+2)a_{k+2}+a_{k-2} = 0 \text{ for all } k \geq 1.\]
Solving for $a_{k+2}$ in the second equation shows that
\begin{align*}
a_2 &= 0 \\
a_{k+2} &= -\frac{1}{(k+1)(k+2)}a_{k-1} \text{ for }  k \geq 1.
\end{align*}
These last equations are called \emph{recurrence relations} and allow us to write every coefficient of $y$ in terms of $a_0$ and $a_1$. For example, $k=1$ shows that $a_3 = \frac{1}{(2)(3)} a_0 = \frac{1}{6}a_0$. We can continue in this way to obtain the first 10 coefficients in terms of $a_0$ and $a_1$:
\begin{align*}
a_3 &= \frac{1}{(2)(3)}a_0 \\
a_4 &= \frac{1}{(3)(4)}a_1 \\
a_5 &= \frac{1}{(4)(5)} a_2 = 0 \\
a_6 &= \frac{1}{(5)(6)} a_3 = \frac{1}{(5)(6)} \left(\frac{1}{(2)(3)} a_0 \right) = \frac{1}{(2)(3)(5)(6)} a_0 \\
a_7 &= \frac{1}{(6)(7)} a_4 = \frac{1}{(6)(7)} \left(\frac{1}{(3)(4)} a_1 \right) = \frac{1}{(3)(4)(6)(7)} a_1 \\
a_8 &= \frac{1}{(7)(8)} a_5 = 0 \\
a_9 &= \frac{1}{(8)(9)} a_6 = \frac{1}{(8)(9)} \left(\frac{1}{(2)(3)(5)(6)} a_0 \right) = \frac{1}{(2)(3)(5)(6)(8)(9)} a_0 \\
a_{10} &= \frac{1}{(9)(10)} a_7 = \frac{1}{(9)(10)} \left(\frac{1}{(3)(4)(6)(7)} a_1 \right) = \frac{1}{(3)(4)(6)(7)(9)(10)} a_1.
\end{align*}

It may not be obvious, but there is a pattern.
\begin{itemize}
\item All of the terms involving $a_2$ are 0. These terms are $a_2$, $a_5$, $a_8$, etc. The subscripts of these terms are all of the form $3k+2$. So $a_{3k+2} = 0$ for every positive integer $k$.
\item The terms that involve $a_0$ have the form $a_3$, $a_6$, $a_{9}$, etc. and are all of the form $a_{3k}$ for positive integers $k$. The pattern in the denominators of the coefficient for $a_{3k}$ is $(2)(3)(5)(6) \cdots (3k-1)(3k)$. So
    \[a_{3k} = \frac{1}{(2)(3)(5)(6) \cdots (3k-1)(3k)} a_0 \text{ for } k \geq 1.\]
\item The terms that involve $a_1$ have the form $a_4$, $a_7$, $a_{10}$, etc. and are all of the form $a_{3k+1}$ for positive integers $k$. The pattern in the denominators of the coefficient for $a_{3k+1}$ is $(3)(4)(6)(7) \cdots (3k)(3k+1)$. So
    \[a_{3k+1} = \frac{1}{(3)(4)(6)(7) \cdots (3k)(3k+1)} a_1 \text{ for } k \geq 1.\]
\end{itemize}

\end{exerciseSolution}

    \item Use the previous part to conclude that the general solution to the Airy equation (\ref{eq:PA8.6_Airy_equation}) is
\begin{equation*} %\label{eq:PA8.6_Airy_solution}
y = a_0\left( 1+\sum_{k=1}^{\infty} \frac{x^{3k}}{(2)(3)(5)(6) \cdots (3k-1)(3k)} \right) + a_1 \left( x + \sum_{k=1}^{\infty} \frac{x^{3k+1}}{(3)(4)(6)(7) \cdots (3k)(3k+1)} \right).
\end{equation*}
Any values for $a_0$ and $a_1$ then determine a specific solution that we can approximate as closely as we like using this series solution.

\begin{exerciseSolution}

We can write our solution $y$ in three pieces as
\[y = \sum_{k = 0}^{\infty} a_kx^k = \sum_{k=0}^{\infty} a_{3k}x^{3k} + \sum_{k=0}^{\infty} a_{3k+1}x^{3k+1} + \sum_{k=0}^{\infty} a_{3k+2}x^{3k+2}\]
and so our results from the previous part of this exercise show that
\[y = a_0\left( 1+\sum_{k=1}^{\infty} \frac{x^{3k}}{(2)(3)(5)(6) \cdots (3k-1)(3k)} \right) + a_1 \left( x + \sum_{k=1}^{\infty} \frac{x^{3k+1}}{(3)(4)(6)(7) \cdots (3k)(3k+1)} \right).\]

\end{exerciseSolution}

    \ea

\begin{exerciseSolution}

\item A more complicated differential equation is one that governs the motion of a pendulum. Mechanics tells us that the motion of a pendulum of length $L$ with mass $m$ is governed by the differential equation
\[mL \frac{d^2 \theta}{d t^2} = -mg \sin(\theta)\]
where $t$ is time and $\theta$ is the angle the pendulum makes with its vertical axis.

We do not have the machinery to solve this differential equation exactly, but we can approximate a solution as closely as we like by assuming that a solution $\theta$ has a Taylor series expansion about the origin. In other words, assume that
\[\theta = \sum_{n=0}^{\infty} a_nt_n,\]
where the coefficients are undetermined. Then
\begin{align*}
\theta &= \sum_{n=0}^{\infty} a_nt_n \\
\frac{d \theta}{dt} &= \sum_{n=1}^{\infty} na_nt_{n-1} \\
\frac{d^2 \theta}{dt^2} &= \sum_{n=2}^{\infty} (n-1)na_nt_{n-2} \\
\frac{d^3 \theta}{dt^3} &= \sum_{n=3}^{\infty} (n-2)(n-1)na_nt_{n-3} \\
\vdots &= \vdots
\end{align*}
from the series expansion and
\begin{align*}
\frac{d^2 \theta}{dt^2} &= -\frac{g}{L} \sin\left(\theta\right)  \\
\frac{d^3 \theta}{dt^3} &= -\frac{g}{L}\left[\cos(\theta)\frac{d \theta}{dt}\right]  \\
\frac{d^4 \theta}{dt^4} &= -\frac{g}{L}\left[-\sin(\theta)\left(\frac{d \theta}{dt}\right)^2 + \cos(theta)\frac{d^2 \theta}{dt^2} \right]  \\
\vdots &= \vdots
\end{align*}
If we seek a solution with initial conditions $\theta(0) = \frac{\pi}{6}$ and $\frac{d \theta}{dt}(0) = 0$, then we have
\begin{align*}
\theta(0) &= a_0 = \frac{\pi}{6} \\
\frac{d \theta}{dt}(0) &= a_1 = 0  \\
\frac{d^2 \theta}{dt^2}(0) &= 2a_2 = -\frac{g}{L} \sin\left(\frac{\pi}{6}\right) = -\frac{g}{2L}  \\
\frac{d^3 \theta}{dt^3}(0) &= 6a_3 = -\frac{g}{L}\cos(\theta(0))\frac{d \theta}{dt}(0) = 0  \\
\frac{d^4 \theta}{dt^4}(0) &= 24a_4 = -\frac{g}{L}\left[-\sin\left(\frac{\pi}{6}\right)\left(\frac{d \theta}{dt}(0)\right)^2 + \cos\left(\frac{\pi}{6}\right)\frac{d^2 \theta}{dt^2}(0) \right] = -\frac{g}{L}\left[\frac{\sqrt{3}}{2}\left(-\frac{g}{2L}\right)\right]  \\
\vdots &= \vdots
\end{align*}
This gives us
\[a_0 = \frac{\pi}{6}, \ \ a_1 = 0, \ \ a_2 = -\frac{g}{4L}, \ \ a_3 = 0, \ \ a_4 = \frac{\sqrt{3} g^2}{96L^2}\]
and an approximation of $\theta$ as
\[\theta \approx \frac{\pi}{6} - \frac{g}{4L}t^2 + \frac{\sqrt{3}g^2}{96L^2} t^4.\]
We could keep going to determine more and more terms in this Taylor series for $\theta$, but the point is that we can use series like this one to approximate functions. The difference between this type of problem and our use of Taylor series is that to determine the terms in a Taylor series, we need to know the function from which we derive the series. In this differential equation example, we didn't know the function $\theta$, be we could calculate terms in a series expansion for $\theta$ because we had enough information about $\theta$.

\end{exerciseSolution}

\end{exercises}

\afterexercises
