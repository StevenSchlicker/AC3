\begin{exercises}
\item Conditionally convergent series converge very slowly. As an example, consider the famous formula\footnote{We will derive this formula in upcoming work.}
\begin{equation} \label{Ex:8.4_pi}
\frac{\pi}{4} = 1 - \frac{1}{3} + \frac{1}{5} - \frac{1}{7} + \cdots = \sum_{k=0}^{\infty} (-1)^{k} \frac{1}{2k+1}.
\end{equation}
 In theory, the partial sums of this series could be used to approximate $\pi$.
    \ba
    \item Show that the series in (\ref{Ex:8.4_pi}) converges conditionally.

    \item Let $S_n$ be the $n$th partial sum of the series in (\ref{Ex:8.4_pi}). Calculate the error in approximating $\frac{\pi}{4}$ with $S_{100}$ and explain why this is not a very good approximation.

    \item Determine the number of terms it would take in the series (\ref{Ex:8.4_pi}) to approximate $\frac{\pi}{4}$ to 10 decimal places. (The fact that it takes such a large number of terms to obtain even a modest degree of accuracy is why we say that conditionally convergent series converge very slowly.)

\begin{exerciseSolution}

In this case we have
\[|S_n - \sum_{k=0}^{\infty} (-1)^{k} \frac{1}{2k+1}| < a_{n+1} = \frac{1}{2n+1}.\]
To have $a_{n+1} < 10^{-10}$ would require
\begin{align*}
\frac{1}{2n+1} &< 10^{-10} \\
2n+1 &> 10^{10} \\
n &> \frac{10^{10}-1}{2} \\
n &> 4,999,999,999.5.
\end{align*}
\end{exerciseSolution}

    \ea

\item We have shown that if $\sum (-1)^{k+1} a_k$ is a convergent alternating series, then the sum $S$ of the series lies between any two consecutive partial sums $S_n$. This suggests that the average $\frac{S_n+S_{n+1}}{2}$
    is a better approximation to $S$ than is $S_n$.
    \ba
    \item Show that $\frac{S_n+S_{n+1}}{2} = S_n + \frac{1}{2}(-1)^{n+2} a_{n+1}$.

\begin{exerciseSolution}

Since $S_{n+1} = S_n + (-1)^{n+2}a_{n+1}$ we have
\begin{align*}
\frac{S_n+S_{n+1}}{2} &= \frac{S_n+S_{n} + (-1)^{n+2}a_{n+1}}{2} \\
    &= \frac{2S_{n} + (-1)^{n+2}a_{n+1}}{2} \\
    &= S_n + \frac{1}{2}(-1)^{n+2}a_{n+1}.
\end{align*}

\end{exerciseSolution}

    \item Use this revised approximation in (a) with $n = 20$ to approximate $\ln(2)$ given that
    \[\ln(2) = \sum_{k=1}^{\infty} (-1)^{k+1} \frac{1}{k}.\]
    Compare this to the approximation using just $S_{20}$. For your convenience, $S_{20} = \frac{155685007}{232792560}$.

    \ea

\item In this exercise, we examine one of the conditions of the Alternating Series Test. Consider the alternating series
\[1 - 1 + \frac{1}{2} - \frac{1}{4} + \frac{1}{3} - \frac{1}{9} + \frac{1}{4} - \frac{1}{16} + \cdots,\]
where the terms are selected alternately from the sequences $\left\{\frac{1}{n}\right\}$ and $\left\{-\frac{1}{n^2}\right\}$.
    \ba
    \item Explain why the $n$th term of the given series converges to 0 as $n$ goes to infinity. 

\begin{exerciseSolution}

Since both sequences $\left\{\frac{1}{n}\right\}$ and $\left\{-\frac{1}{n^2}\right\}$ converge to 0 as $n$ goes to infinity, we see that the sequence of $n$th terms of our series converge to 0 as $n$ goes to infinity.    

\end{exerciseSolution}
 
    \item Rewrite the given series by grouping terms in the following manner:
\[(1 - 1) + \left(\frac{1}{2} - \frac{1}{4}\right) + \left(\frac{1}{3} - \frac{1}{9}\right) + \left(\frac{1}{4} - \frac{1}{16}\right) + \cdots. \]
Use this regrouping to determine if the series converges or diverges. 

\begin{exerciseSolution}
Notice that $\frac{1}{k} - \frac{1}{k^2} = \frac{k-1}{k^2}$. So our series can be rewritten as
\[\sum_{k=1}^{\infty} \frac{k-1}{k^2}.\]
Since 
\[\lim_{k \to \infty} \frac{ \frac{k-1}{k^2} }{ \frac{1}{k} } = \lim_{k \to \infty} \frac{k^2-k}{k^2} = 1,\]
the Limit Comparison Test shows that the series $\sum_{k=1}^{\infty} \frac{k-1}{k^2}$ behaves like the harmonic series $\sum_{k=1}^{\infty} \frac{1}{k}$. Since the harmonic series diverges, it follows that the series $\sum_{k=1}^{\infty} \frac{k-1}{k^2}$ diverges as well. 

\end{exerciseSolution}

    \item Explain why the condition that the sequence $\{a_n\}$ \emph{decreases} to a limit of 0 is included in the Alternating Series Test. 
    
        \ea

\item Conditionally convergent series exhibit interesting and unexpected behavior. In this exercise we examine the conditionally convergent alternating harmonic series $\ds \sum_{k=1}^{\infty} \frac{(-1)^{k+1}}{k}$ and discover that addition is not commutative for conditionally convergent series. We will also encounter Riemann's Theorem concerning rearrangements of conditionally convergent series. Before we begin, we reminder ourselves that
    \[\sum_{k=1}^{\infty} \frac{(-1)^{k+1}}{k} = \ln(2),\]
    a fact which will be verified in a later section.
\ba
\item First we make a quick analysis of the positive and negative terms of the alternating harmonic series.
    \begin{itemize}
    \item[(i)] Show that the series $\ds \sum_{k=1}^{\infty} \frac{1}{2k}$ diverges.


    \item[(ii)] Show that the series $\ds \sum_{k=1}^{\infty} \frac{1}{2k+1}$ diverges.


    \item[(iii)] Based on the results of the previous parts of this exercise, what can we say about the sums $\ds \sum_{k=C}^{\infty} \frac{1}{2k}$ and $\ds \sum_{k=C}^{\infty} \frac{1}{2k+1}$ for any positive integer $C$? Be specific in your explanation.

    \end{itemize}

\item Recall addition of real numbers is commutative; that is
\[a + b = b + a\]
for any real numbers $a$ and $b$. This property is valid for any sum of finitely many terms, but does this property extend when we add infinitely many terms together?

 The answer is no, and something even more odd happens. Riemann's Theorem (after the nineteenth-century mathematician Georg Friedrich Bernhard Riemann) states that a conditionally convergent series can be rearranged to converge to \emph{any} prescribed sum. More specifically, this means that if we choose any real number $S$, we can rearrange the terms of the alternating harmonic series $\ds \sum_{k=1}^{\infty} \frac{(-1)^{k+1}}{k}$ so that the sum is $S$. To understand how Riemann's Theorem works, let's assume for the moment that the number $S$ we want our rearrangement to converge to is positive. Our job is to find a way to order the sum of terms of the alternating harmonic series to converge to $S$.
    \begin{itemize}
    \item[(i)] Explain how we know that, regardless of the value of $S$, we can find a partial sum $P_1$ of the positive terms
    \[P_1 = \sum_{k=1}^{n_1} \frac{1}{2k+1} = 1 + \frac{1}{3} + \frac{1}{5} + \cdots + \frac{1}{2n_1+1}\]
    of the positive terms of the alternating harmonic series that equals or exceeds $S$. Let
    \[S_1 = P_1.\]

    \item[(ii)] Explain how we know that, regardless of the value of $S_1$, we can find a partial sum $N_1$
    \[N_1 = -\sum_{k=1}^{m_1} \frac{1}{2k} = -\frac{1}{2} - \frac{1}{4} - \frac{1}{6} - \cdots - \frac{1}{2m_1}\]
    so that
    \[S_2 = S_1 + N_1 \leq S.\]

    \item[(iii)] Explain how we know that, regardless of the value of $S_2$, we can find a partial sum $P_2$
    \[P_2 = \sum_{k=n_1+1}^{n_2} \frac{1}{2k+1} = \frac{1}{2(n_1+1)+1} + \frac{1}{2(n_1+2)+1} + \cdots + \frac{1}{2n_2+1}\]
    of the remaining positive terms of the alternating harmonic series so that
    \[S_3 = S_2 + P_2 \geq S.\]

    \item[(iv)] Explain how we know that, regardless of the value of $S_3$, we can find a partial sum
    \[N_2 = -\sum_{k=m_1+1}^{m_2} \frac{1}{2k} = -\frac{1}{2(m_1+1)} - \frac{1}{2(m_1+2)} - \cdots - \frac{1}{2m_2}\]
     of the remaining negative terms of the alternating harmonic series so that
    \[S_4 = S_3 + N_2 \leq S.\]

    \item[(v)] Explain why we can continue this process indefinitely and find a sequence $\{S_n\}$ whose terms are partial sums of a rearrangement of the terms in the alternating harmonic series so that $\ds \lim_{n \to \infty} S_n = S$.


    \end{itemize}

 \ea





\end{exercises}


\afterexercises
