\begin{exercises} 
  \item  The mass of a radioactive sample decays at a rate that is
    proportional to its mass. 
    \ba
  \item Express this fact as a differential equation for the mass
    $M(t)$ using $k$ for the constant of proportionality.
    \item If the initial mass is $M_0$, find an expression for the
      mass $M(t)$.
    \item The {\em half-life} of the sample is the amount of time
      required for half of the mass to decay.  Knowing that the
      half-life of Carbon-14 is 5730 years, find the value of $k$ for
      a sample of Carbon-14.
    \item How long does it take for a sample of Carbon-14 to be
      reduced to one-quarter its original mass?
    \item Carbon-14 naturally occurs in our environment; any
      living organism takes in Carbon-14 when it eats and breathes.  Upon
      dying, however, the organism no longer takes in Carbon-14. 
      
      Suppose that you find remnants of a pre-historic firepit.  By
      analyzing the charred wood in the pit, you determine that the
      amount of Carbon-14 is only 30\% of the amount in living trees.
      Estimate the age of the firepit.\footnote{This approach is the basic idea behind radiocarbon dating.}

      
    \ea

  \item Consider the initial value problem
    
    $$  \frac{dy}{dt} = -\frac ty, \ y(0) = 8$$
    

    \ba
    \item Find the solution of the initial value problem and sketch its
      graph.

    \item For what values of $t$ is the solution defined?  

    \item What is the value of $y$ at the last time that the
      solution is defined? 

    \item By looking at the differential equation, explain why we
      should not expect to find solutions with the value of $y$ you noted in (c).

      \ea

  \item  Suppose that a cylindrical water tank with a hole in the
    bottom is filled with water.  The water, of course, will leak out
    and the height of the water will decrease.  Let $h(t)$ denote the
    height of the water.  A physical principle called {\em Torricelli's
      Law} implies that the height decreases at a rate proportional to
    the square root of the height.

    \ba
    \item Express this fact using $k$ as the constant of
      proportionality.  
    \item Suppose you have two tanks, one with $k=1$ and another with
      $k=10$.  What physical differences would you expect to find?
    \item Suppose you have a tank for which the height decreases at 20
      inches per minute when the water is filled to a depth of 100
      inches.  Find the value of $k$.  
    \item Solve the initial value problem for the tank in part (c).
    \item Find $h(t)$, the height of the water in the tank and sketch
      a graph of it.
    \item How long does it take for the water to run out of the tank?
    \item Is the solution that you found valid for all time $t$?  If
      so, explain how you know this.  If not, explain why not.
    \ea

  \item The {\em Gompertz equation} is a model that is used to
    describe the growth of certain populations.  Suppose that $P(t)$
    is the population of some organism and that
    $$
    \frac{dP}{dt} = -P\ln\left(\frac P3\right) = -P(\ln P - \ln 3).
    $$

    \ba
    \item Sketch a slope field for $P(t)$ over the range $0\leq P\leq
      6$.

    \item Identify any equilibrium solutions and determine whether
      they are stable or unstable.

    \item Find the population $P(t)$ assuming that $P(0) = 1$ and sketch
      its graph.  What happens to $P(t)$ after a very long time?

    \item Find the population $P(t)$ assuming that $P(0) = 6$ and sketch
      its graph.  What happens to $P(t)$ after a very long time?

    \item Verify that the long-term behavior of your solutions agrees
      with what you predicted by looking at the slope field.

      \ea
        
\end{exercises}
\afterexercises
