\begin{exercises} 
  \item The probability of a transistor failing between months $t = a$ and $t = b$ is given by the probability density function $p(t) = ce^{-ct}$, where this formula is valid for all $t \ge 0$ and $p(t) = 0$ for all $t < 0$.
\ba
	\item If the probability of the transistor failing in the first 6 months is 0.1, what is the value of $c$?
	\item Suppose that for a different transistor, the value of $c$ is known to be $c = 0.05$.   For this $c$ value, find the mean time it takes for a transistor to fail.
	\item For the same $c$-value ($c = 0.05$), find the median time it takes for a transistor to fail. 
\ea

\item  A probability density function is given by $p(x) = \frac{1}{4a^2} (x-a)^2$, where $a > 0$ and the formula is valid of $0 \le x \le a$, while $p(x) = 0$ if $x < 0$ or $x > a$.  Suppose that $p(x)$ models the amount of time, in minutes, a customer waits at a drive-in bank teller.
\ba
	\item What is the value of $a$?
	\item For the (similar, but different) pdf $p(x) = \frac{3}{125} (x-5)^2$ (which is valid for $0 \le x \le 5$, otherwise $p(x) = 0$), find the mean wait time.
	\item Determine the median wait time.
	\item Find a formula that does not involve an integral for the cdf, $P$, that corresponds to this pdf.
\ea	
  
\end{exercises}
\afterexercises
