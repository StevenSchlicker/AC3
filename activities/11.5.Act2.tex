\begin{activity} \label{A:11.5.2} Most polar graphing devices\footnote{You can use your calculator in POL mode, or a web applet such as \url{http://webspace.ship.edu/msrenault/ggb/polar_grapher.html}} can plot curves in polar coordinates of the form $r = f(\theta)$. Use such a device to complete this activity.
	\ba
	\item Before plotting the polar curve $r=1$, think about what shape it should have, in light of how $r$ is connected to $x$ and $y$. Then use appropriate technology to draw the graph and test your intuition.

	\item The equation $\theta = 1$ does not define $r$ as a function of $\theta$, so we can't graph this equation on many polar plotters. What do you think the graph of the polar curve $\theta = 1$ looks like? Why?
	
	\item Before plotting the polar curve $r = \theta$, what do you think the graph looks like? Why? Use technology to plot the curve and compare your intuition.

         \item What about the curve $r = \sin(\theta)$?  After plotting this curve, experiment with others of your choosing and think about why the curves look the way they do.	
    	
	\ea

\end{activity}
\begin{smallhint}

\end{smallhint}
\begin{bighint}

\end{bighint}
\begin{activitySolution}
	\ba
	\item Since $r$ represents a distance from the origin, any curve with a constant value of $r$ should be a circle, centered at the origin, with radius $r$. 

	\item The set of points with a constant value of $\theta$ all make the same angle with the positive $x$-axis. This set of points should then form a line through the origin making an angle $\theta$ with the positive $x$-axis.  
		
	\item As $\theta$ increases, so does the value of $r$. Thus, as the point $(r,r)$ rotates around the origin, its distance from the origin also increases in a uniform manner. The set of these points should be a spiral, spiraling away from the origin as it rotates counterclockwise around the origin. 	

    \item As $\theta$ increases, the values of $r$ will oscillate between $-1$ and $1$. When $r$ is negative, we reflect around the origin. So the resulting curve should look like a circle in the first and second quadrants. Note that $r=\sin(\theta)$ can also be represented as $r^2 = r\sin(\theta)$. So in rectangular coordinates the curve $r=\sin(\theta)$ has equation $x^2+y^2 = y$, or $x^2 + \left(y-\frac{1}{2}\right)^2 = \frac{1}{4}$. This is a circle  centered at $\left(0,frac{1}{2}\right)$ with radius $\frac{1}{2}$.   

    	
	\ea
\end{activitySolution}
\aftera
