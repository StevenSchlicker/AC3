\begin{activity} \label{A:10.3.2} Let $f = f(x,y)$ be a function of the independent variables $x$ and $y$.
    \ba
	\item Run the animation in Figure \ref{F:10.3.PD_xx}. Explain as best
you can what the second order unmixed partial derivative
\[\frac{\partial^2 f}{\partial x^2} = \frac{\partial}{\partial x} \frac{\partial f}{\partial x}\]
tells us about the graph of $f$ by first considering what the first order partial derivative $\frac{\partial f}{\partial x}$ represents, then
describing what happens to this first order partial derivative as we increase $x$.



    \item Run the animation in Figure \ref{F:10.3.PD_yy}. Explain as best you can what the second order unmixed partial derivative $\ds
        \frac{\partial^2 f}{\partial y^2}$ tells us about the graph of $f$.



  \ea

\end{activity}
\begin{smallhint}

\end{smallhint}
\begin{bighint}

\end{bighint}
\begin{activitySolution}
\ba
\item The second order unmixed partial derivative $f_{xx}$ tells us how the first order partial derivative $f_x$ changes for each unit increase in $x$ while keeping $y$ constant. 

Since the first order partial derivative $f_x$ gives the slope of the tangent lines to the surface in the $x$-direction, the second order unmixed partial $f_{xx}$ tells us how the slopes of the tangent lines to the $x$ trace of $f$ are changing as we increase $x$. We can interpret this as telling us the   the concavity of the trace in the $x$ direction. 

\item The second order unmixed partial derivative $f_{yy}$ tells us how the first order partial derivative $f_y$ changes for each unit increase in $y$ while keeping $y$ constant. 

Since the first order partial derivative $f_y$ gives the slope of the tangent lines to the surface in the $y$-direction, the second order unmixed partial $f_{yy}$ tells us how the slopes of the tangent lines to the $y$ trace of $f$ are changing as we increase $y$. We can interpret this as telling us the   the concavity of the trace in the $y$ direction. 
\ea
\end{activitySolution}
\aftera
