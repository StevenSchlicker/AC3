\begin{activity} \label{A:10.2.7} Often we are certain graphical information about a function instead of a rule. We can use that information to approximate partial derivatives. For example, suppose that we are given a contour plot of the kinetic energy function (as in Figure \ref{F:10.2.pd_contour}) instead of a formula. Use this contour plot to approximate $f_x(4,5)$ and $f_y(4,5)$ as best you can. Compare to your calculations from Activities \ref{A:10.2.5} and \ref{A:10.2.6}.


\end{activity}
\begin{smallhint}

\end{smallhint}
\begin{bighint}

\end{bighint}
\begin{activitySolution}
To approximate $f_x(4,5)$, we will use a symmetric difference quotient like we did in single variable calculus. That is,
\[f_x(4,5) \approx \frac{f(4+h,5)-f(4-h,5)}{2h}\]
for a suitable value of $h$. We can approximate the values of $f$ on the contour plot for $h = 1$, giving us 
\[f_x(4,5) \approx  \frac{f(5,5)-f(3,5)}{2} \approx \frac{60-40}{2} = 10.\]
Similarly,
\[f_y(4,5) \approx  \frac{f(4,6)-f(4,4)}{2} \approx \frac{65-30}{2} = 17.5.\]
These values compare reasonably well to the values of $f_x(4,5) = 12.5$ and $f_y(4,5) = 20$ obtained in Activities \ref{A:10.2.5} and \ref{A:10.2.6}. 
\end{activitySolution}
\aftera
