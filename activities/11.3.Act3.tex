\begin{activity} \label{A:11.3.3} Consider the iterated integral $\ds \int_{x=0}^{x=4} \int_{y=x/2}^{y=2} e^{y^2} \, dy \, dx$.

\ba
  \item Explain why we cannot antidifferentiate $e^{y^2}$ with respect to $y$, and thus are unable to evaluate the iterated integral $\ds  \int_{x=0}^{x=4} \int_{y=x/2}^{y=2} e^{y^2} \, dy \, dx$ using the Fundamental Theorem of Calculus. 
  \item Sketch the region of integration, $D$, so that $\ds \iint_D e^{y^2} \, dA = \int_{x=0}^{x=4} \int_{y=x/2}^{y=2} e^{y^2} \, dy \, dx.$
  \item Rewrite the given iterated integral in the opposite order, using $dA = dx \, dy$.
  \item Use the Fundamental Theorem of Calculus to evaluate the iterated integral you developed in (d).  Write one sentence to explain the meaning of the value you found.
  \item What is the important lesson this activity offers regarding the order in which we set up an iterated integral?
\ea

\end{activity}
\begin{smallhint}

\end{smallhint}
\begin{bighint}

\end{bighint}
\begin{activitySolution}
\ba
\item No substitution will work to integrate $e^{y^2}$ because of the square in the exponent. Using parts only makes the integral more complicated, and we have no other techniques to apply to this integrand.

\item The domain $D$ of integration is shown below. 
%Shift up 0.5 to leave space for the x-labels
\begin{center}
\setlength{\unitlength}{1.0cm}
\begin{picture}(4.5,3.0)
\linethickness{0.35mm}
\put(0,0.5){\vector(1,0){4.5}}
\put(0,0.5){\vector(0,1){2.5}}
\put(0,2.5){\line(1,0){4}}
\color{red}
\put(0,0.5){\line(2,1){4.0}}
\color{black}
\put(4.4,0.2){$x$}
\put(-0.3,2.9){$y$}
\put(1,0.4){\line(0,1){0.2}}
\put(2,0.4){\line(0,1){0.2}}
\put(3,0.4){\line(0,1){0.2}}
\put(4,0.4){\line(0,1){0.2}}
\put(-0.1,1.5){\line(1,0){0.2}}
\put(-0.1,2.5){\line(1,0){0.2}}
\put(3.9,0){$4$}
\put(-0.4,2.4){$2$}
\put(2.5,1.5){$y=\frac{x}{2}$}

\color{blue}
\linethickness{0.35mm}
\put(0,1.0){\line(1,0){1.0}}
\put(0,2.0){\line(1,0){3.0}}
\end{picture}
\end{center}

\item When we first integrate with respect to $x$, we can see that the cross sections go from $x=0$ to $x = 2y$. The limits on $y$ are then $0$ to $2$, giving us
\[\iint_D e^{y^2} \, d = \int_0^2 \int_{0}^{2y} e^{y^2} \, dx \, dy.\]

\item Evaluating the iterated integral yields 
\begin{align*}
\int_0^4 \int_{0}^{x/2} e^{y^2} \, dy \, dx &= \int_0^2 \int_{0}^{2y} e^{y^2} \, dx \, dy \\
	&= \int_0^2 \left. xe^{y^2} \right|_{0}^{2y} \, dy \\
	&= \int_0^2 2ye^{y^2} \, dy \\
	&= \left. e^{y^2} \right|_0^2 \\
	&= e^4-1.
\end{align*}
Since $e^{y^2} > 0$ on $D$, this integral tells us the volume of the solid bounded above by the surface $z=e^{y^2}$ and below by the region $D$. 

\item We need to be flexible when setting up iterated integrals -- sometimes one order of integration is very difficult (or impossible) while the other order is easier. So it is important to practice changing the order of integration in iterated integrals. 

\ea

\end{activitySolution}

\aftera
