\begin{activity} \label{A:9.6.1} The same curve can be represented
  with different parameterizations. Use your calculator,\footnote{If
    you have a graphing calculator you can draw graphs of
    vector-valued functions in $\R^2$ using the parametric mode (often
    found in the MODE menu).} Wolfram$\mid$Alpha, or some other
  graphing device\footnote{e.g.,
    \url{http://webspace.ship.edu/msrenault/ggb/parametric_grapher.html}}
  to plot the curves generated by the following vector-valued functions. Compare and contrast the graphs -- explain
  how they are alike and how they are different.  \ba
    \item $\vr(t) = \langle \sin(t), \cos(t) \rangle$



    \item $\vr(t) = \langle \sin(2t), \cos(2t) \rangle$



    \item $\vr(t) = \langle \cos(t+\pi), \sin(t+\pi) \rangle$



    \ea

\end{activity}
\begin{smallhint}

\end{smallhint}
\begin{bighint}

\end{bighint}
\begin{activitySolution}
    \ba
    \item The graph of $\vr(t) = \langle \sin(t), \cos(t) \rangle$ is the unit circle. The fact that $\vr(0) = \langle 0,1 \rangle$ and $\vr\left(\frac{\pi}{2}\right) = \langle 1,0 \rangle$ indicates that this parameterization traces out the unit circle in a clockwise manner.  The parameterization also traces out each quadrant using the same length interval, e.g., the graph is in the first quadrant when $t$ is between 0 and $\frac{\pi}{2}$, and in the fourth quadrant for $t$ between $\frac{\pi}{2}$ and $\pi$), which illustrates that the parameterization traces out the circle in a uniform manner.
    \item The graph of $\vr(t) = \langle \sin(2t), \cos(2t) \rangle$ is again the unit circle. In this case, we have $\vr(0) = \langle 0,1 \rangle$ and $\vr\left(\frac{\pi}{4}\right) = \langle 1,0 \rangle$, which indicates that this parameterization traces out the unit circle in a clockwise manner.  The parameterization also traces out each quadrant using the same length interval, e.g., the graph is in the first quadrant when $t$ is between 0 and $\frac{\pi}{4}$, and in the fourth quadrant for $t$ between $\frac{\pi}{4}$ and $\frac{\pi}{2}$) illustrates that the parameterization traces out the circle in a uniform manner. However, the $t$ intervals in which this parameterization lies in a given quadrant are half as long as those in part (a), which tells us that this parameterization traces out the unit circle in half the time, or twice as fast as the parameterization in part (a). 
    \item The graph of $\vr(t) = \langle \cos(t+\pi), \sin(t+\pi) \rangle$ is again the unit circle. In this case, we have $\vr(0) = \langle -1,0 \rangle$ and $\vr\left(\frac{\pi}{2}\right) = \langle 0,-1 \rangle$, which indicates that this parameterization traces out the unit circle in a counterclockwise manner.  The parameterization also traces out each quadrant using the same length interval, e.g., the graph is in the third quadrant when $t$ is between 0 and $\frac{\pi}{2}$, and in the fourth quadrant for $t$ between $\frac{\pi}{4}$ and $\frac{\pi}{2}$), which illustrates that the parameterization traces out the circle in a uniform manner and at the same rate as the parameterization in part (a) (but in the opposite direction). 
    \ea
\end{activitySolution}
\aftera
