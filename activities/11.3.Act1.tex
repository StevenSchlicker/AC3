\begin{activity} \label{A:11.3.1}  Consider the double integral $\ds \iint_D (4-x-2y) \, dA$, where $D$ is the triangular region with vertices (0,0), (4,0), and (0,2).
		\ba
		\item Write the given integral as an iterated integral of the form $\ds \iint_D (4-x-2y) \, dy \, dx$. Draw a labeled picture of $D$ with relevant cross sections.

		\item Write the given integral as an iterated integral of the form $\ds \iint_D (4-x-2y) \, dx \, dy$. Draw a labeled picture of $D$ with relevant cross sections.

		\item Evaluate the two iterated integrals from (a) and (b), and verify that they produce the same value. Give at least one interpretation of the meaning of your result.

	\ea

\end{activity}
\begin{smallhint}

\end{smallhint}
\begin{bighint}

\end{bighint}
\begin{activitySolution}
\ba
		\item Integrating with respect to $y$ first means we take cross sections of $D$ parallel to the $y$-axis. As the figure below illustrates, the lower end of any cross section is always on the $x$-axis (with $y=0$) and the upper limit is on the line joining $(4,0)$ and $(0,2)$. The limits on $y$ need to written in terms of $x$, so the upper limit is $y = 2-\frac{1}{2}x$. These cross sections run from $x=0$ to $x=4$, so as an iterated integral we have 
\[\iint_D \left(\frac{3}{4}\right)(4-x-2y) \, dy \, dx = \int_{0}^{4} \int_{0}^{2-(1/2)x} \left(\frac{3}{4}\right)(4-x-2y) \, dy \, dx.\]

%Shift up 0.5 to leave space for the x-labels
\begin{center}
\setlength{\unitlength}{1.0cm}
\begin{picture}(4.5,3.0)
\linethickness{0.35mm}
\put(0,0.5){\vector(1,0){4.5}}
\put(0,0.5){\vector(0,1){2.5}}
\color{red}
\put(0,2.5){\line(2,-1){4.0}}
\color{black}
\put(4.4,0.2){$x$}
\put(-0.3,2.9){$y$}
\put(1,0.4){\line(0,1){0.2}}
\put(2,0.4){\line(0,1){0.2}}
\put(3,0.4){\line(0,1){0.2}}
\put(4,0.4){\line(0,1){0.2}}
\put(-0.1,1.5){\line(1,0){0.2}}
\put(-0.1,2.5){\line(1,0){0.2}}
\put(3.9,0){$4$}
\put(-0.4,2.4){$2$}
\put(2,1.7){$y=2-\frac{1}{2}x$}

\color{blue}
\linethickness{0.35mm}
\put(0.2,0.5){\line(0,1){1.9}}
\put(1.2,0.5){\line(0,1){1.4}}
\put(2.2,0.5){\line(0,1){0.9}}
\put(3.2,0.5){\line(0,1){0.4}}
\end{picture}
\end{center}


		\item Integrating with respect to $x$ first means we take cross sections of $D$ parallel to the $x$-axis. As the figure below illustrates, the lower end of any cross section is always on the $y$-axis (with $x=0$) and the upper limit is on the line joining $(4,0)$ and $(0,2)$. The limits on $x$ need to written in terms of $y$, so the upper limit is $x = 4-2y$. These cross sections run from $y=0$ to $y=2$, so as an iterated integral we have 
\[\iint_D \left(\frac{3}{4}\right)(4-x-2y) \, dx \, dy = \int_{0}^{2} \int_{0}^{4-2y} \left(\frac{3}{4}\right)(4-x-2y) \, dx \, dy.\]

\begin{center}
\setlength{\unitlength}{1.0cm}
\begin{picture}(4.5,3.0)
\linethickness{0.35mm}
\put(0,0.5){\vector(1,0){4.5}}
\put(0,0.5){\vector(0,1){2.5}}
\color{red}
\put(0,2.5){\line(2,-1){4.0}}
\color{black}
\put(4.4,0.2){$x$}
\put(-0.3,2.9){$y$}
\put(1,0.4){\line(0,1){0.2}}
\put(2,0.4){\line(0,1){0.2}}
\put(3,0.4){\line(0,1){0.2}}
\put(4,0.4){\line(0,1){0.2}}
\put(-0.1,1.5){\line(1,0){0.2}}
\put(-0.1,2.5){\line(1,0){0.2}}
\put(3.9,0){$4$}
\put(-0.4,2.4){$2$}
\put(2,1.7){$x=4-2y$}

\color{blue}
\linethickness{0.35mm}
\put(0,0.9){\line(1,0){3.2}}
\put(0,1.3){\line(1,0){2.4}}
\put(0,2.1){\line(1,0){0.8}}
\end{picture}
\end{center}


\item Evaluating the iterated integral with respect to $y$ then $x$ gives us  
\begin{align*}
\iint_D \left(\frac{3}{4}\right)(4-x-2y) \, dy \, dx &= \int_{0}^{4} \int_{0}^{2-(1/2)x} \left(\frac{3}{4}\right)(4-x-2y) \, dy \, dx \\
	&= \frac{3}{4} \int_{0}^{4} \left. \left[(4-x)y-y^2\right] \right|_{0}^{2-(1/2)x} \, dx \\
	&= \frac{3}{4} \int_{0}^{4}  \left[(4-x)\left(2-\frac{1}{2}x\right)-\left(2-\frac{1}{2}x\right)^2\right] \, dx \\
	&= \frac{3}{4} \int_{0}^{4}  \left[4-2x+\frac{1}{4}x^2\right] \, dx \\
	&= \frac{3}{4} \left. \left[4x-x^2+\frac{1}{12}x^3\right] \right|{0}^{4} \\
	&= \frac{3}{4} \left(\frac{16}{3}\right) \\
	&= 4.
\end{align*}
Evaluating the iterated integral with respect to $y$ then $x$ gives us 
\begin{align*}
\iint_D \left(\frac{3}{4}\right)(4-x-2y) \, dx \, dy &= \int_{0}^{2} \int_{0}^{4-2y} \left(\frac{3}{4}\right)(4-x-2y) \, dx \, dy \\
	&= \frac{3}{4} \int_{0}^{2}  \left. \left[4x-\frac{x^2}{2}-2yx \right] \right|_{0}^{4-2y} \, dy \\
	&= \frac{3}{4} \int_{0}^{2}   \left[4(4-2y)-\frac{(4-2y)^2}{2}-2y(4-2y) \right] \, dy \\
	&= \frac{3}{4} \int_{0}^{2}   \left[8-8y+2y^2 \right] \, dy \\
	&= \frac{3}{4} \left. \left[8y-4y^2+\frac{2}{3}y^3 \right] \right|_{0}^{2}  \\
	&= \frac{3}{4} \left(\frac{16}{3}\right)  \\
	&= 4.
\end{align*}
Since $f$ is non-negative on the domain $D$, these two iterated integrals tell us the volume of the solid bounded above by the graph of $f(x,y) = \frac{3}{4}(4-x-2y)$ and below by the triangular region $D$. 

\ea
\end{activitySolution}
\aftera
