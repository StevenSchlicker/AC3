\begin{activity} \label{A:9.5.9} 
  \ba
\item Write the equation of the plane $p_1$ passing through the point $(0,
  2, 4)$ and perpendicular to the vector $\vn=\langle 2, -1,
  1\rangle$.

\item Is the point $(2, 0, 2)$ on the plane $p_1$?

\item Write the equation of the plane $p_2$ that is parallel to $p_1$
  and passing through the point $(3, 0, 4)$.  
	
\item Write the parametric description of the line $l$ passing through the
  point $(2,0,2)$ and perpendicular to the plane $p_3$ described the
  equation $x+2y-2z = 7$.

\item Find the point at which $l$ intersects the plane $p_3$.
	
	
	\ea


\end{activity}
\begin{smallhint}

\end{smallhint}
\begin{bighint}

\end{bighint}
\begin{activitySolution}
\ba
\item The plane has equation $2x-(y-2)+(z-4)=0$. 


\item Since $2(2)-(0-2)+(2-4) \neq 0$, the point point $(2, 0, 2)$ is not on the plane $p_1$.

\item Parallel planes will have parallel normal vectors, so an equation of the plane $p_2$ that is parallel to $p_1$
  and passing through the point $(3, 0, 4)$ is $2(x-3)-y+(z-4)=0$.   
	
\item A direction vector for $l$ will be a normal vector for $p_3$, or $\langle 1,2,-2\rangle$. So a parametric description of the line $l$ is 
\[x(t) = 2+t \ \ y(t) = 2t \ \ z(t) = 2-2t.\]

\item When $l$ intersects $p_3$ we will have $(2+t)+2(2t)-2(2-2t) = 7$. This happens when $t=1$, which gives the point of intersection as $(3,2,0)$. 
\ea
\end{activitySolution}
\aftera
