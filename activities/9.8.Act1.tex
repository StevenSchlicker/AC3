\begin{activity} \label{A:9.8.1} Here we calculate the arc length of two familiar curves.
\ba
	\item Use Equation~(\ref{eq:9.8.arclength_2}) to calculate the circumference of a circle of radius $r$.
	\item Find the exact length of the spiral defined by $\vr(t) = \langle \cos(t), \sin(t), t \rangle$ on the interval $[0,2\pi]$. 
\ea	
\end{activity}
\begin{smallhint}

\end{smallhint}
\begin{bighint}

\end{bighint}
\begin{activitySolution}
\ba
	\item We can parameterize a circle of radius $r$ centered at the origin by $\vr(t) = \langle \cos(t), \sin(t) \rangle$ for $t$ in the interval $[0, 2\pi]$. The circumference of this circle is the length of curve defined by $r$, or
\[\int_0^{2\pi} \lvert \vr'(t) \rvert \, dt = \int_0^{2\pi} \sqrt{(-r\sin(t))^2 + (r\cos(t))^2} \, dt = \int_0^{2\pi} r \, dt = 2 \pi r\]
as expected.
	\item The length of curve defined by $r$ on the interval $[0,2 \pi]$ is 
\[\int_0^{2\pi} \lvert \vr'(t) \rvert \, dt = \int_0^{2\pi} \sqrt{(-\sin(t))^2 + (\cos(t))^2 + 1^2} \, dt = \int_0^{2\pi} \sqrt{2} \, dt = 2 \pi \sqrt{2}.\]
\ea
\end{activitySolution}

%%%%%%%%%%

\aftera
