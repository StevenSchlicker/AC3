\begin{activity} \label{A:10.7.3} Find the critical points of each of the
  following functions. Then, using appropriate technology (e.g.,
  Wolfram$|$Alpha or CalcPlot3D\footnote{at
    \url{http://web.monroecc.edu/manila/webfiles/calcNSF/JavaCode/CalcPlot3D.htm}}),
  plot the graphs of the surfaces near each critical value and compare the graph to your work.
   
	\ba
	\item $f(x,y) = 2+x^2+y^2$
	
	\item $f(x,y) = 2 + x^2 - y^2$
	
	\item $f(x,y) = 2x-x^2-\frac{1}{4}y^2$
	
	\item $f(x,y) = |x| + |y|$
	
        \item $f(x,y) = 2xy - 4x + 2y - 3$.
	
	

	\ea
\end{activity}
\begin{smallhint}

\end{smallhint}
\begin{bighint}

\end{bighint}
\begin{activitySolution}
\ba
\item We determine the points in the domain of $f$ where $\nabla f = \vzero$ or $\nabla f$ does not exist. In this example, 
\[\nabla f = \langle 2x, 2y \rangle,\]
and so $\nabla f = \vzero$ when $x = 0$ and $y = 0$. Thus, $(0,0)$ is the only critical point of $f$. The graph of $f$ is a paraboloid that opens in the positive $z$-direction, so we should expect only one critical point (at which $f$ has an absolute minimum value).

\item For this function,
\[\nabla f = \left\langle 2-2x, -\frac{1}{2}y \right\rangle.\]
So $\nabla f = \vzero$ when $2-2x = 0$ and $y = 0$, or at the point $(1,0)$. Thus, $(1,0)$ is the only critical point of $f$. The graph of $f$ is a paraboloid that opens in the negative $z$-direction, so we should expect only one critical point (at which $f$ has an absolute maximum value).

\item Recall that if $g(x) = |x|$, then $g'(x) = \begin{cases} 1 &\text{if } x > 1 \\ -1 &\text{if } x < 1 \end{cases}$ and $g'(0)$ does not exist. In this case we will never have $\nabla f = \vzero$, but $\nabla f$ will fail to exist when $x=0$ or $y=0$. This gives us infinitely many critical point -- those points of the form $(0,y)$ or $(x,0)$. The graph of $f$ has sharp folds along the $x$-axis or $y$-axis, so we should expect that $f$ is not differentiable at those points. The graph of $f$ also indicates that $f$ has an absolute minimum value at the point $(0,0)$. 
\ea
\end{activitySolution}
\aftera
