\begin{activity} \label{A:10.2.12} The wind chill, as frequently
  reported, is a measure of how cold it feels outside when
  the wind is blowing.  In Table \ref{T:10.2.wind.chill}, the wind
  chill $w$, measured in degrees Fahrenheit, is a function of the wind speed $v$, measured in miles per hour, and
  the ambient air temperature $T$, also measured in degrees
  Fahrenheit.  We thus view $w$ as being of the form $w = w(v, T)$.

\begin{table}[ht] 
  \begin{center}
    \begin{tabular}{|c||c|c|c|c|c|c|c|c|c|c|c|}
      \hline
      $v \backslash T$  
         &-30  &-25 &-20 &-15 &-10 &-5  &0   &5   &10  &15  &20  \\
      \hhline{|=|=|=|=|=|=|=|=|=|=|=|=|}
      5  &-46	&-40 &-34 &-28 &-22 &-16 &-11 &-5 &1 &7 &13  \\
      \hline
      10 &-53	&-47 &-41 &-35 &-28 &-22 &-16 &-10 &-4 &3 &9   \\
      \hline
      15 &-58	&-51 &-45 &-39 &-32 &-26 &-19 &-13 &-7 &0 &6  \\
      \hline
      20 &-61	&-55 &-48 &-42 &-35 &-29 &-22 &-15 &-9 &-2 &4  \\
      \hline
      25 &-64	&-58 &-51 &-44 &-37 &-31 &-24 &-17 &-11 &-4 &3 \\
      \hline
      30 &-67	&-60 &-53 &-46 &-39 &-33 &-26 &-19 &-12 &-5 &1 \\
      \hline
      35 &-69	&-62 &-55 &-48 &-41 &-34 &-27 &-21 &-14 &-7 &0 \\
      \hline
      40 &-71	&-64 &-57 &-50 &-43 &-36 &-29 &-22 &-15 &-8 &-1 \\
      \hline
    \end{tabular}
    \caption{Wind chill as a function of wind speed and temperature.}
    \label{T:10.2.wind.chill}
  \end{center}
\end{table}
%\begin{table}[ht]
%  \begin{center}
%    \begin{tabular}{|c||c|c|c|c|c|c|c|c|c|c|c|}
%      \hline
%      $v \backslash T$  
%         &-30  &-25 &-20 &-15 &-10 &-5  &0   &5   &10  &15  &20  \\
%      \hhline{|=|=|=|=|=|=|=|=|=|=|=|=|}
%      5  &-35  &-31 &-26 &-20 &-15 &-11 &-6  &1   &7   &12  &16  \\
%      \hline
%      10 &-58  &-52 &-45 &-38 &-31 &-27 &-22 &-15 &-9  &-2  &2   \\
%      \hline
%      15 &-70  &-65 &-60 &-51 &-45 &-40 &-33 &-25 &-18 &-11 &-6  \\
%      \hline
%      20 &-81  &-76 &-68 &-60 &-52 &-46 &-40 &-32 &-24 &-17 &-9  \\
%      \hline
%      25 &-89  &-83 &-75 &-67 &-58 &-52 &-45 &-37 &-29 &-22 &-15 \\
%      \hline
%      30 &-94  &-87 &-78 &-70 &-63 &-56 &-49 &-41 &-33 &-26 &-18 \\
%      \hline
%      35 &-98  &-90 &-83 &-72 &-67 &-60 &-52 &-43 &-35 &-27 &-20 \\
%      \hline
%      40 &-101 &-94 &-87 &-76 &-69 &-62 &-54 &-45 &-36 &-29 &-22 \\
%      \hline
%    \end{tabular}
%    \caption{Wind chill as a function of temperature and wind speed}
%    \label{T:10.2.wind.chill}
%  \end{center}
%\end{table}

\ba
\item Estimate the partial derivative $w_v(20,-10)$.  What are the
  units on this quantity and what does it mean?
\item Estimate the partial derivative $w_T(20,-10)$.  What are the
  units on this quantity and what does it mean?
\item Use your results to estimate the wind chill $w(18, -10)$.
\item Use your results to estimate the wind chill $w(20, -12)$.
\item Use your results to estimate the wind chill $w(18, -12)$.
\ea

\end{activity}

\begin{activitySolution}
\ba
\item Recall that 
\[w_v(v,T) = \lim_{h \to 0} \frac{w(v+h,T) - w(v,T)}{h},\]
which gives us the slope of the tangent line to the trace in the $v$ direction. We can approximate this slope with the slope of a secant line, choosing points equally spaced on both sides of $(v,T)$. In other words,
\[w_v(v,T) \approx \frac{w(v+h,T) - w(v-h,T)}{2h},\]
for small values of $h$. This is the symmetric difference quotient from calculus 1. 
The data in our table allows us to use $h=5$ for wind speed, so 
\[w_v(20,-10) \approx \frac{w(25,-10)-w(15,-10)}{10} = \frac{-37-(-32)}{10} = -\frac{1}{2}\]
degrees Fahrenheit per mile per hour. So for every one mile per hour increase in the wind speed from 20 miles per hour, while holding the air temperature constant at $-10^{\circ}F$, the wind chill decreases by approximately 0.5 $^{\circ}F$. 
\item We approximate $w_T(20,-10)$ in the same way, using $h=5$ again. So 
\[w_T(20,-10) \approx \frac{w(20,-5)-w(20,-15)}{10} = \frac{-29-(-42)}{10} = 1.3\]
degrees Fahrenheit per degree Fahrenheit. So for every one degree Fahrenheit increase in the wind temperature $-10^{\circ}F$, while holding the wind speed constant at 20 miles per hour, the wind chill increases by approximately 1.3 $^{\circ}F$.  
\item This is just like a linearization from calculus 1 along the trace with $T=-10$. We found the slope of the tangent line to the $T=-10$ trace of $w$ at the point $(20,-10)$ to be $-0.5$, so  
\[w(18,-10) \approx  w(20,-10) - 0.5(18-20) = -35 + 1 = -34.\]
\item This is just like a linearization from calculus 1 along the trace with $v=20$. We found the slope of the tangent line to the $v=20$ trace of $w$ at the point $(20,-10)$ to be $1.3$, so  
\[w(20,-12) \approx  w(20,-10) + 1.3(-12-(-10)) = -35 - 2.6 = -37.6.\]
\item We might expect that the total change in $w$ will be calculated by using the changes in $w$ that arise from changing both the $v$ and $T$ coordinates. In other words,
\[w(18,-12) \approx w(20,-10) - 0.5(18-20) + 1.3(-12-(-10)) = -35 + 1 - 2.6 = -36.6.\]
\ea
\end{activitySolution}

\aftera
