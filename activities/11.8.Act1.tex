\begin{activity} \label{A:11.8.1} In this activity, we graph some surfaces using cylindrical coordinates. To improve your intuition and test your understanding, you should first think about what each graph should look like before you plot it using technology.\footnote{e.g., \url{http://www.math.uri.edu/~bkaskosz/flashmo/cylin/} -- to plot $r=2$, set $r$ to 2, $\theta$ to $s$, and $z$ to $t$ -- to plot $\theta = \pi/3$, set $\theta = \pi/3$, $r=s$, and $z=t$, for example. Thanks to Barbara Kaskosz of URI and the Flash and Math team.}
    \ba
    \item Plot the graph of the cylindrical equation $r=2$, where we restrict the values of $\theta$ and $z$ to the intervals  $0 \leq \theta \leq 2\pi$ and $0 \leq z \leq 2$. What familiar shape does the resulting surface take?  How does this example suggest that we call these coordinates \emph{cylindrical coordinates}?



    \item Plot the graph of the cylindrical equation $\theta=2$, where we restrict the other variables to the values $0 \leq r \leq 2$ and $0 \leq z \leq 2$. What familiar surface results?



    \item Plot the graph of the cylindrical equation $z=2$, using $0 \leq \theta \leq 2\pi$ and $0 \leq r \leq 2$. What does this surface look like?



    \item Plot the graph of the cylindrical equation $z=r$, where $0 \leq \theta \leq 2\pi$ and $0 \leq r \leq 2$. What familiar surface results?



    \item Plot the graph of the cylindrical equation $z= \theta$ for $0 \leq \theta \leq 4 \pi$. What does this surface look like?



    \ea

\end{activity}
\begin{smallhint}

\end{smallhint}
\begin{bighint}

\end{bighint}
\begin{activitySolution}
    \ba
    \item In polar coordinates, the graph of $r=2$ is a circle centered at the origin with radius 2. Extending this to cylindrical coordinates, we obtain a cylinder centered at the origin of radius 2, with base on the $xy$-plane and height 2. 

    \item In polar coordinates, the graph of $\theta = 2$ is a line through the origin making an angle of 2 radians with the positive $x$-axis. Extending this to cylindrical coordinates, the graph of $\theta=2$ is a plane through the origin perpendicular to the $xy$-plane making an angle of 2 radians with the $xz$-plane. 

    \item The graph of $z=2$ is just a plane parallel to the $xy$-plane at a distance 2 above the $xy$-plane. With $0 \leq \theta \leq 2\pi$ and $0 \leq r \leq 2$, the graph of $z=2$ is just a disk of radius 2 centered at $(0,0,2)$ within the plane $z=2$. 


    \item For a fixed value of $r$, the graph of $z=r$ and $0 \leq \theta \leq 2\pi$ is a circle of radius $r$ in the plane $z=r$. As $r$ increases the radii increase, so we will obtain a cone with base at the origin, opening up with height $2$. 

    \item For a fixed $\theta$, the graph of $z=\theta$ is a line in the $z = \theta$ plane making an angle of $\theta$ with the $xz$-plane. Allowing $\theta$ to vary will produce a set of lines, spiraling upward so that the result looks like a bit of an auger. 

    \ea
\end{activitySolution}
\aftera