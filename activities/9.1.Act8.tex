\begin{activity} \label{A:9.1.8}
   On the topographical map of the Porcupine Mountains in Figure \ref{F:9.1.porcupine},

   \ba
    \item identify the highest and lowest points you can find;
    \item from a point of your choice, determine a path of steepest ascent that leads to the highest point;
    \item from that same initial point, determine the least steep path that leads to the highest point.

     \ea

\end{activity}
\begin{smallhint}

\end{smallhint}
\begin{bighint}
\ba
\item The best you can do is search the map for the highest and lowest elevations.
\item Closely spaced contours indicate a sharp increase or decrease in elevation.
\item Widely spaced contours indicate a slow increase or decrease in elevation. 
\ea
\end{bighint}
\begin{activitySolution}
\ba
\item Summit Peak appears to be the highest point on this map at an elevation of around 593. Near the top left corner of the map there is a contour at an elevation of around 200 that seems to be the lowest on the map. 
\item The contours to the west of Summit Peak appear to be the most closely spaced, indicating the path of steepest ascent to the peak.
\item The contours to the southwest of Summit Peak appear to be the most widely spaced, indicating the path of most gentle ascent to the peak.
\ea
\end{activitySolution}

%\newpage
%\begin{landscape}
%\begin{figure}[ht]
%\begin{center}
%\resizebox{!}{4.0in}{\includegraphics[trim=40cm 40cm 40cm 40cm, clip]{figures/1_1_porcupine_2}} %trim=left bottom right top, need clip if not animation
%\caption{Contour map of the Porcupine Mountains.}
%\label{F:1.1.porcupine}
%\end{center}
%\end{figure}
%\end{landscape}
%\newpage

\aftera 