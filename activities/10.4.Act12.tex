\begin{activity} \label{A:10.4.12} The questions in this activity explore the differential in several different contexts.
\ba
\item Suppose that the elevation of a landscape is given by the
  function $h$, where we additionally know that $h(3,1) = 4.35$, $h_x(3,1) = 0.27$, and $h_y(3,1) = -0.19$.  Assume that $x$ and $y$ are measured in miles in the easterly and northerly directions, respectively, from some base point $(0,0)$.

  Your GPS device says that you are currently at the point $(3,1)$.
  However, you know that the coordinates are only accurate to within
  $0.2$ units; that is, $dx = \Delta x = 0.2$ and $dy= \Delta y =
  0.2$.  Estimate the uncertainty in your elevation using differentials.

\item The pressure, volume, and temperature of an ideal gas are
  related by the equation 
  $$
  P= P(T,V) = 8.31 T/V,
  $$
  where $P$ is measured in kilopascals, $V$ in liters, and $T$ in
  kelvin.  Find the pressure when the volume is 12 liters and the
  temperature is 310 K.  Use differentials to estimate the change
  in the pressure when the volume increases to 12.3 liters and the
  temperature decreases to 305 K.

\item Refer to Table \ref{T:10.4.wind.chill}, the table of
  values of the wind chill $w(v,T)$, in degrees Fahrenheit, as a
  function of temperature, also in degrees Fahrenheit, and 
  wind speed, in miles per hour.  
  
  Suppose your anemometer says the wind is blowing at $25$ miles per hour and your thermometer shows a reading of $-15^\circ$ degrees.
  However, you know your thermometer is only accurate to within
  $2^\circ$ degrees and your anemometer is only accurate to within $3$
  miles per hour.  What is the wind chill based on your measurements?
  Estimate the uncertainty in your measurement of the wind chill.

\ea

\end{activity}

\begin{activitySolution}
\ba
\item The error is approximated by the differential 
\[\Delta h \approx dh = h_x(3,1)~dx + h_y(3,1)~dy = 0.27(0.2) - 0.19(0.2) = 0.016.\]

\item First note that $P(310,12) = 214.675$. To find the differential, we need both $P_T(310,12)$ and $P_V(310,12)$. Our differentiation rules give us 
\begin{align*}
P_T(T,V) &= 8.31\frac{1}{V} \\
P_V(T,V) &= -8.31\frac{T}{V^2}.
\end{align*}
So $P_T(310,12) = 0.6925$ and $P_V(310,12) \approx -17.8896$. We have $dT = -5$ and $dV = 0.3$, and the change in pressure is 
\[\Delta P \approx dP = P_T(310,12)~dT + P_V(310,12)~dV \approx 0.6925(-5) - 17.8896(0.3) = -8.82938.\]
Therefore, the pressure decreases by about 8.82938 kilopascals.

\item First note that 
\begin{align*}
w_v(25,-15) &\approx \frac{w(30,-15)-w(20,-15)}{10} = \frac{-46-(-42)}{10} = -0.4 \\
w_T(25,-15) &\approx \frac{w(25,-10)-w(25,-20)}{10} = \frac{-37-(-51)}{10} = 1.4.
\end{align*}
The wind chill is $w(25,-15) = -44^{\circ}F$. But since $dT = 2$ and $dv = 3$, our wind chill measurement has an error of
\[\Delta w \approx dw = -0.4(3) + 1.4(2) = 1.6^{\circ}F.\]


\ea
\end{activitySolution}

\aftera
