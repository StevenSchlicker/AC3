 \begin{activity} \label{A:10.2.1} If $f = f(x,y)$ is a function of the two independent variables $x$ and $y$ and we hold $x$ constant and allow $y$ to vary, then we can differentiate the resulting function of $y$ to obtain what we call the partial derivative of $f$ with respect to $y$. Complete the following definition:

\begin{quote} The partial derivative of $f$ with respect to $y$ is
\[\frac{\partial f}{\partial y}(x,y) = f_y(x,y) = ...\]
\end{quote}



\end{activity}
\begin{smallhint}

\end{smallhint}
\begin{bighint}

\end{bighint}
\begin{activitySolution}
The partial derivative of $f$ with respect to $y$ is
\[\frac{\partial f}{\partial y}(x,y) = f_y(x,y) = \lim_{h \to 0} \frac{f(x,y+h)-f(x,y)}{h}.\]
\end{activitySolution}
\aftera
