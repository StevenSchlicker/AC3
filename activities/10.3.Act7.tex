\begin{activity} \label{A:10.3.7}  Suppose we have a function $f(x,y)$
  with the values given in Table \ref{T:10.3.activity.1}.

\begin{table}[ht]
  \begin{center}
    \begin{tabular}{|c||c|c|}
      \hline
      $y \backslash x$ & 5 & 5.1 \\
      \hline
      \hline
      3 & 2.7 & 3.2 \\
      \hline
      3.1 & 2.5 & 2.8 \\
      \hline
    \end{tabular}
    \caption{Values for $f(x,y)$.}
    \label{T:10.3.activity.1}
  \end{center}
\end{table}

\ba
\item Use this table of values to estimate the partial $f_{xy}(5,
  3)$.  

\item In the same way, estimate the partial $f_{yx}(5,3)$. 

\item Similarly, let's consider the function $g(x,y)$ with
  values given in Table \ref{T:10.3.activity.2}.  Estimate the partial
  $g_{xy}(5,3)$ in terms of $a$, $b$, $c$, and $d$.

\begin{table}[ht]
  \begin{center}
    \begin{tabular}{|c||c|c|}
      \hline
      $y \backslash x$ & 5 & 5.1 \\
      \hline
      \hline
      3 & a & b \\
      \hline
      3.1 & c & d \\
      \hline
    \end{tabular}
    \caption{Values for $g(x,y)$.}
    \label{T:10.3.activity.2}
  \end{center}
\end{table}

\item Also, estimate the partial $g_{yx}(5,3)$ in terms of $a$, $b$,
  $c$, and $d$.

\item Compare the results of the last two parts of this activity.
  How do they reflect Clairaut's theorem?

\ea
\end{activity}
\aftera
