\begin{activity} \label{A:11.8.7} In this activity, we graph some surfaces using spherical coordinates. To improve your intuition and test your understanding, you should first think about what each graph should look like before you plot it using technology.\footnote{e.g., \url{http://www.flashandmath.com/mathlets/multicalc/paramsphere/surf_graph_sphere.html} -- to plot $\rho=2$, set $\rho$ to 2, $\theta$ to $s$, and $\phi$ to $t$, for example. Thanks to Barbara Kaskosz of URI and the Flash and Math team.}
    \ba
    \item Plot the graph of $\rho = 1$, where $\theta$ and $\phi$ are restricted to the intervals  $0 \leq \theta \leq 2\pi$ and $0 \leq \phi \leq \pi$. What is the resulting surface? How does this particular example demonstrate the reason for the name of this coordinate system?

    \item Plot the graph of $\phi = \frac{\pi}{3}$, where $\rho$ and $\theta$ are restricted to the intervals $0 \leq \rho \leq 1$ and $0 \leq \theta \leq 2\pi$. What familiar surface results?

    \item Plot the graph of $\theta = \frac{\pi}{6}$, for $0 \leq \rho \leq 1$ and $0 \leq \phi \leq \pi$. What familiar shape arises?

    \item Plot the graph of $\rho = \theta$, for $0 \leq \phi \leq \pi$ and $0 \leq \theta \leq 2 \pi$. How does the resulting surface appear?



    \ea

\end{activity}
\begin{smallhint}

\end{smallhint}
\begin{bighint}

\end{bighint}
\begin{activitySolution}
    \ba
    \item Since $\rho$ measures the distance from the origin to a point, the graph of $\rho = 1$ is the set of all points that are a distance 1 from the origin. This makes $\rho=1$ the equation of the sphere of radius 1 centered at the origin. 

    \item All points with $\phi = \frac{\pi}{3}$ are at a fixed angle from the positive $z$-axis. The set of such points with $0 \leq \rho \leq 1$ is a cone making an angle of $\frac{\pi}{3}$ with the positive $z$-axis having a slant height of 1. 

    \item As in cylindrical coordinates, the graph of $\theta = 2$ is a plane through the origin perpendicular to the $xy$-plane making an angle of 2 radians with the $xz$-plane. With $0 \leq \rho \leq 1$ and $0 \leq \phi \leq \pi$, we obtain a semicircle or radius 1 centered at the origin in this plane.


    \item For a fixed value of $\theta$, the graph of $\rho=\theta$ is an intersection of a sphere of radius $\theta$ centered at the origin with a plane through the origin. If $0 \leq \phi \leq \pi$, then we get a semicircle. As $\theta$ increases from $0$ to $2\pi$, the radii of the semicircles increase, leaving us with a surface that looks like a snail shell. 

    \ea
\end{activitySolution}
\aftera
