\begin{activity} \label{A:10.2.4} Let $f(x,y) = \sin(x)e^{-y}$. A graph of $f$ on the domain $[0,\pi] \times [-2,0]$ is shown in Figure \ref{F:10.2.PD_y}.
    \ba
	\item Run the animation in Figure \ref{F:10.2.PD_y}. Pick a frame and explain as best you can, and in as much detail as you can, what the difference quotient
\[\frac{f(a,b+h) - f(a,b)}{h}\]
represents, geometrically, for some appropriate value of $h$.



    \item Now explain in as much detail as you can what you think the partial derivative
\[\lim_{h \to 0} \frac{f(a,b+h) - f(a,b)}{h}\]
represents, geometrically, in Figure \ref{F:10.2.PD_y}.



  \ea

\end{activity}
\begin{smallhint}

\end{smallhint}
\begin{bighint}

\end{bighint}
\begin{activitySolution}
\ba
\item As the animation progresses, the secant lines to the surface in the $y$-direction connecting the points $(a,b,f(a,b))$ and $(a,b+h,f(a,b+h))$ are shown. The difference quotient $\frac{f(a,b+h) - f(a,b)}{h}$ gives the change in the function values divided by the change in $y$, or the slopes of these secant lines.
\item Just as in single variable calculus, as $h$ goes to $0$ the slopes of the secant lines become the slope of the tangent line. So the partial derivative $f_y(a,b) = \lim_{h \to 0} \frac{f(a,b+h) - f(a,b)}{h}$ is the slope of the line tangent to the surface defined by $f$ in the $y$-direction at the point $(a,b, f(a,b))$. 
\ea
\end{activitySolution}
\aftera
