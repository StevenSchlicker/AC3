\begin{activity} \label{A:11.4.6} A firm manufactures smoke detectors. Two components for the detectors come from different suppliers -- one in Michigan and one in Ohio. The company studies these components for their reliability and their data suggests that if $x$ is the life span (in years) of a randomly chosen component from the Michigan supplier and $y$ the life span (in years) of a randomly chosen component from the Ohio supplier, then the joint probability density function $f$ might be given by
\[f(x,y) = e^{-x} e^{-y}.\]
    \ba
    \item Theoretically, the components might last forever, so the domain $D$ of the function $f$ is the set $D$ of all $(x,y)$ such that $x \ge 0$ and $y \ge 0$. To show that $f$ is a probability density function on $D$ we need to demonstrate that
        \[\int \int_D f(x,y) \, dA = 1,\]
        or that
        \[\int_0^{\infty} \int_0^{\infty} f(x,y) \, dy \, dx = 1.\]
        Use your knowledge of improper integrals to verify that $f$ is indeed a probability density function.

    \item Assume that the smoke detector fails only if both of the supplied components fail. To determine the probability that a randomly selected detector will fail within one year, we will need to determine the probability that the life span of each component is between 0 and 1 years. Set up an appropriate iterated integral, and evaluate the integral to determine the probability.
    
    \item What is the probability that a randomly chosen smoke detector will fail between years 3 and 7?
    
    \item Suppose that the manufacturer determines that one of the components is more likely to fail than the other, and hence conjectures that the probability density function is instead $f(x,y) = K e^{-x} e^{-2y}.$  What is the value of $K$?

    \ea

\end{activity}
\begin{smallhint}

\end{smallhint}
\begin{bighint}

\end{bighint}
\begin{activitySolution}
\ba
\item We evaluate the double integral to determine if $f$ is a probability density function:
\begin{align*}
\int_0^{\infty} \int_0^{\infty} f(x,y) \, dy \, dx &= \int_0^{\infty} \int_0^{\infty} e^{-x} e^{-y} \, dy \, dx \\
	&= \int_0^{\infty} e^{-x} \left( \int_0^{\infty}  e^{-y} \, dy \right) \, dx \\
	&= \int_0^{\infty} e^{-x} \left(\lim_{b \to \infty} \int_0^{b}  e^{-y} \, dy \right) \, dx \\
	&= \int_0^{\infty} e^{-x} \left(\lim_{b \to \infty} \left. -e^{-y} \right|_0^{b} \right) \, dx \\
	&= \int_0^{\infty} e^{-x} \left(\lim_{b \to \infty} -e^{-b}+1 \right) \, dx \\
	&= \int_0^{\infty} e^{-x} (1) \, dx \\
	&= \lim_{b \to \infty} \int_0^{b}  e^{-x} \, dx \\
	&= \lim_{b \to \infty} \left. -e^{-x} \right|_0^{b} \\
	&= \lim_{b \to \infty} -e^{-b}+1 \\
	&= 1.
\end{align*}
So $f$ is a probability density function.
	

\item We need both components to have a life span of 1 year or less, so the probability that a randomly selected detector will fail within one year is
\begin{align*}
\int_0^1 \int_0^1 f(x,y) \, dy \, dx &= \int_0^1 \int_0^1 e^{-x} e^{-y} \, dy \, dx \\
	&= \int_0^1  -e^{-x} e^{-y} \bigm|_{y=0}^1 \, dx \\
	&= \int_0^1  e^{-x} \left(1-\frac{1}{e}\right) \, dx \\
	&= \left(1-\frac{1}{e}\right) (-e^{-x})\bigm|_{x=0}^1 \\
	&= \left(1-\frac{1}{e}\right)^2.
\end{align*}
So the probability that a randomly selected detector will fail within one year is approximately 40\%.

\item We need both components to have a life span of between 3 and 7 years, so the probability that a randomly selected detector will fail within 3 to 7 years is
\begin{align*}
\int_3^7 \int_3^7 f(x,y) \, dy \, dx &= \int_3^7 \int_3^7 e^{-x} e^{-y} \, dy \, dx \\
	&= \int_3^7  -e^{-x} e^{-y} \bigm|_{y=3}^7 \, dx \\
	&= \int_3^7  e^{-x} \left(\frac{1}{e^3}-\frac{1}{e^7}\right) \, dx \\
	&= \left(\frac{1}{e^3}-\frac{1}{e^7}\right) (-e^{-x})\bigm|_{x=3}^7 \\
	&= \left(\frac{1}{e^3}-\frac{1}{e^7}\right)^2.
\end{align*}
This amounts to approximately 0.24\%. 

\item To find $K$ we need to determine when 
\[\int_0^{\infty} \int_0^{\infty} f(x,y) \, dy \, dx = 1.\]
We evaluate the iterated integral to determine the value of $K$:
\begin{align*}
\int_0^{\infty} \int_0^{\infty} f(x,y) \, dy \, dx &= \int_0^{\infty} \int_0^{\infty} Ke^{-x} e^{-2y} \, dy \, dx \\
	&= K\int_0^{\infty} e^{-x} \left( \int_0^{\infty}  e^{-2y} \, dy \right) \, dx \\
	&= K\int_0^{\infty} e^{-x} \left(\lim_{b \to \infty} \int_0^{b}  e^{-2y} \, dy \right) \, dx \\
	&= K\int_0^{\infty} e^{-x} \left(\lim_{b \to \infty} \left. -\frac{1}{2}e^{-2y} \right|_0^{b} \right) \, dx \\
	&= K\int_0^{\infty} e^{-x} \left(\lim_{b \to \infty} -\frac{1}{2}(e^{-2b}-1) \right) \, dx \\
	&= \frac{K}{2}\int_0^{\infty} e^{-x}  \, dx \\
	&= \frac{K}{2}\lim_{b \to \infty} \int_0^{b}  e^{-x} \, dx \\
	&= \frac{K}{2}\lim_{b \to \infty} \left. -e^{-x} \right|_0^{b} \\
	&= \frac{K}{2}\lim_{b \to \infty} -e^{-b}+1 \\
	&= \frac{K}{2}.
\end{align*}
So we need to have $K = 2$ for $f$ to be a probability density function.

\ea
\end{activitySolution}
\aftera


