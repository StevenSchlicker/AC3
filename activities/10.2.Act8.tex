\begin{activity} \label{A:10.2.8} Often we are given numerical information about a function instead of a rule. We can use that information to approximate partial derivatives. For example, suppose that we are given a table of values of the kinetic energy function (as in Table \ref{T:10.2.PD_KE}) instead of a formula. Use this table of data to approximate $f_x(4,5)$ and $f_y(4,5)$ as best you can. Compare to your calculations from Activities \ref{A:10.2.5} and \ref{A:10.2.6}.

\end{activity}
\begin{smallhint}

\end{smallhint}
\begin{bighint}

\end{bighint}
\begin{activitySolution}
To approximate $f_x(4,5)$, we will use a symmetric difference quotient like we did in single variable calculus. That is,
\[f_x(4,5) \approx \frac{f(4+h,5)-f(4-h,5)}{2h}\]
for a suitable value of $h$. We can use the values of $f$ from the data set with $h = 0.1$, giving us 
\[f_x(4,5) \approx  \frac{f(4.1,5)-f(3.9,5)}{0.2} = \frac{51.25-48.75}{0.2} = 12.5.\]
Similarly, we use $h = 0.2$ to obtain 
\[f_y(4,5) \approx  \frac{f(4,5.2)-f(4,4.8)}{0.4} \approx \frac{54.08-46.08}{0.4} = 20.\]
These values are the same as the values of $f_x(4,5) = 12.5$ and $f_y(4,5) = 20$ obtained in Activities \ref{A:10.2.5} and \ref{A:10.2.6}. Given the much smaller values of $h$ we were able to use with the data, we should expect better approximations than we obtained through the contour plot. 
\end{activitySolution}
\aftera
