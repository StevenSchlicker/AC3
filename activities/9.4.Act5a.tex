\begin{activity} \label{A:9.4.5}  Suppose 
  $\vu = \langle 2, -1, 0\rangle$ and $\vv = \langle 0, 1, 3\rangle$.
  Use the formula (\ref{E:9.4.cross.def}) for the following.
  \ba
  \item Find the cross product $\vu\times\vv$.  

  \item Find the cross product $\vu\times \vi$.

  \item Find the cross product $\vu\times\vu$.

  \item Evaluate the dot products $\vu\cdot(\vu\times\vv)$ and
    $\vv\cdot(\vu\times\vv)$.  What does this tell you about the
    geometric relationship among $\vu$, $\vv$, and $\vu\times\vv$?

    \ea
\end{activity}

\begin{activitySolution}
\ba
	\item By the formula we have
\[\langle 2, -1, 0\rangle \times \langle 0, 1, 3\rangle = ((-1)(3)-0) \vi - ((2)(3)-0)\vj + ((2)(1)-0)\vk = -3 \vi - 6\vj + 2\vk.\]

	\item By the formula we have
\[\langle 2, -1, 0\rangle \times \langle 1, 0, 0\rangle = (0) \vi - (0)\vj + (0-(-1)(1))\vk = \vk.\] 

	\item By the formula we have
\[\langle 2, -1, 0\rangle \times \langle 2, -1, 0\rangle = (0) \vi - (0)\vj + (-2-(-2))\vk = \vzero.\] 

	\item Using the definition of the dot product we have
	\begin{align*}
	\vu\cdot(\vu\times\vv) &= \langle 2, -1, 0\rangle \cdot \langle -3, -6, 2 \rangle = 0 \\
	\vv\cdot(\vu\times\vv) &= \langle 0, 1, 3\rangle \cdot \langle -3, -6, 2 \rangle = 0.
	\end{align*}
So both $\vu$ and $\vv$ are perpendicular to $\vu\times\vv$. 
	\ea
\end{activitySolution}

\aftera
