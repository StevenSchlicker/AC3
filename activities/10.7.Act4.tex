\begin{activity} \label{A:10.7.4} Return to our revenue function $f(p_1,p_2) = p_1(150 - 2p_1 - p_2) + p_2(200 - p_1 - 3p_2)$ from our Preview Activity. Recall that
\[\frac{\partial f}{\partial p_1} = 150 - 2p_2 - 4p_1 \ \ \text{ and } \ \ \frac{\partial f}{\partial p_2} = 200 - 2p_1 - 6p_2,\]
so the sole critical point of $f$ is $(25,25)$.
	\ba
	\item Find the discriminant of $f$ at its critical point.
	
	
	
	\item Use the Second Derivative Test to classify the critical point of $f$. Does the result confirm what we learned in the Preview Activity?
	
	
	
	\ea
\end{activity}
\begin{smallhint}

\end{smallhint}
\begin{bighint}

\end{bighint}
\begin{activitySolution}
\ba
\item To find the discriminant of $f$ at $(25,25)$ we need the second order partial derivatives of $f$. Now
\[\frac{\partial^2 f}{\partial p_1^2} = -4, \ \  \frac{\partial^2 f}{\partial p_2^2} = -6, \ \ \text{ and } \ \ \frac{\partial^2 f}{\partial p_1 \partial p_2} = -2.\]
So the discriminant of $f$ at $(25,25)$ is
\[D = f_{p_1p_1}(25, 25) f_{p_2p_2}(25, 25) - f_{p_1p_2}(25, 25)^2 = (-4)(-6) - (-2)^2 = 20.\]

\item Since the discriminant at $(25,25)$ is positive and $f_{p_1p_1}(25, 25) < 0$, then $f$ has a relative maximum at $(25, 25)$. This confirms what we saw in our preview activity. 
\ea 
\end{activitySolution}
\aftera
