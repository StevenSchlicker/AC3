\begin{activity} \label{A:10.7.2}Let $f(x,y) = \sin(x)+\cos(y)$. Determine the absolute maximum and minimum values of $f$. At what points do these extreme values occur?
\end{activity}
\begin{smallhint}

\end{smallhint}
\begin{bighint}

\end{bighint}
\begin{activitySolution}
We know that $|\cos(x)| \leq 1$ and $|\sin(y)| \leq 1$, so $f$ cannot have values larger than 2 or less than $-2$. Since 
\[f\left(\frac{\pi}{2},  m \pi\right) = \sin\left(\frac{\pi}{2}\right) + \cos(0) = 2 \ \ \text{ and } \ \ f\left(-\frac{\pi}{2},  \pi \right) = \sin\left(-\frac{\pi}{2}\right) + \cos(\pi) = -2,\]
we see that the absolute maximum value of $f$ is 2 and the absolute minimum value of $f$ is -2. 
\end{activitySolution}
\aftera
