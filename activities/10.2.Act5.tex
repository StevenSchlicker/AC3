\begin{activity} \label{A:10.2.5} Let $f(x,y) = \frac{1}{2}xy^2$ represent the kinetic energy in Joules of an object of mass $x$ in kilograms with velocity $y$ in meters per second. Let $(a,b)$ be the point $(4,5)$ in the domain of $f$.
    \ba
    \item Calculate $f_x(a,b)$.



    \item Explain as best you can in the context of kinetic energy what the partial derivative
\[f_x(a,b) = \lim_{h \to 0} \frac{f(a+h,b) - f(a,b)}{h}\]
tells us about kinetic energy.

    \ea

\end{activity}
\begin{smallhint}

\end{smallhint}
\begin{bighint}

\end{bighint}
\begin{activitySolution}
    \ba
    \item A straightforward calculation shows that 
\[f_x(x,y) = \frac{1}{2}y^2.\]
So 
\[f_x(a,b) = \frac{1}{2}(5^2) = 12.5.\]

    \item The value of $f_x(a,b)$ tells us that when the mass of an object is 4 kg and the velocity of the object is 5 meters per second, the kinetic energy of the object increases by approximately 12.5 Joules for every one kg increase in the mass of the object, if the velocity is kept constant. 

    \ea
\end{activitySolution}
\aftera
