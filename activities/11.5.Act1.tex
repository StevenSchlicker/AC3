\begin{activity} \label{A:11.5.1} \ba
	\item Find the rectangular coordinates of the point whose polar coordinates are $\left(2, -\frac{\pi}{3}\right)$.



	\item Find polar coordinates for the point whose rectangular coordinates are $P=(-1, 1)$. Now find a second pair of polar coordinates for the point $P$. How many different ways are there to represent a point in polar coordinates?
	
	
	
	\ea

\end{activity}
\begin{smallhint}

\end{smallhint}
\begin{bighint}

\end{bighint}
\begin{activitySolution}
\ba
	\item We have $r = 2$ and $\theta = -\frac{\pi}{3}$, so 
\[x = 2 \cos\left(-\frac{\pi}{3}\right) = 1 \ \text{ and } \ y = 2 \sin\left(-\frac{\pi}{3}\right) = -\sqrt{3}.\]

	\item Since $x = -1$ and $y = 1$ we have $r = \sqrt{(-1)^1 + 1^2} = \sqrt{2}$. The point $(-1,1)$ is in the second quadrant, and $\tan(\theta) = -1$, so $\theta = \frac{3\pi}{4}$. We can also use any angle for $\theta$ that is co-terminal with $\frac{3\pi}{4}$, so $\left(\sqrt{2}, \frac{3\pi}{4} + 2k \pi\right)$ is a set of polar coordinates for $(-1,1)$ for any integer $k$. This shows that we can represent the same point in infinitely many different ways in polar coordinates. 
	
	\ea
\end{activitySolution}
\aftera
