\begin{activity} \label{A:10.8.1} A cylindrical soda can holds
  about 355 cc of liquid. In this activity, we want to find the
  dimensions of such a can that will minimize the surface area.  
\ba
    \item What are the variables in this problem? What restriction(s),
      if any, are there on these variables? 

    \item What quantity do we want to optimize in this problem? What
      equation describes the constraint? 

    \item Find $\lambda$ and the values of your variables that satisfy
      Equation (\ref{eq:10.8.Lagrange_ex1}) in the context of this
      problem. 

    \item Determine the dimensions of the pop can that give the
      desired solution to this constrained optimization problem. 


    \ea


\end{activity}
\begin{smallhint}

\end{smallhint}
\begin{bighint}

\end{bighint}
\begin{activitySolution}
\ba
\item Assume the pop can is cylindrical. Let $r$ by the radius and $h$ the height of the can. The volume $V$ of the can is constant at 355 cc of liquid. The radius and height can be any positive numbers as long as the volume is 355.

\item We want to minimize the surface area $A$ of the can, where
\[A(r,h) = 2\pi r^2 + 2\pi r h.\]
The volume of the can is $V(r,h) = \pi r^2 h$, so our constraint is
\[355 = V(r,h) = \pi r^2 h.\]

\item To solve this constrained optimization problem, we need to find the values of $\lambda$, $r$, and $h$ that satisfy
\begin{align*}
\nabla A &= \lambda \nabla V \\
\left\langle 4\pi r + 2\pi h, 2 \pi r \right\rangle &= \lambda \left\langle 2\pi rh, \pi r^2 \right\rangle.
\end{align*}
So we have to solve the equations
\begin{align*}
\pi r^2h &= 355 \\
4\pi r + 2\pi h &= \lambda 2\pi rh \\
2 \pi r &= \lambda \pi r^2.
\end{align*}
Since $r>0$, these equations simplify to
\begin{align*}
2\pi r^2h &= 355 \\
2r + h &= \lambda rh \\
2 &= \lambda r.
\end{align*}
So
\begin{equation} \label{eq:LM_example1}
\lambda = \frac{2}{r} = \frac{2r+h}{rh}.
\end{equation}
Now
\[h = \frac{355}{\pi r^2},\]
so equation (\ref{eq:LM_example1}) becomes
\[\frac{2}{r} = \frac{r+h}{rh} = \frac{2\pi r^3+355}{355r}.\]
Applying some algebra yields
\begin{align*}
\frac{2}{r} &= \frac{2\pi r^3+355}{355r} \\
710 &= 2\pi r^3 + 355 \\
r^3 &= \frac{355}{2\pi} \\
r &= \sqrt[3]{\frac{355}{2\pi}} \\
r &\approx 3.84 \text{ cm}.
\end{align*}
This makes
\[h = \frac{355}{\pi r^2} \approx 7.66 \text{ cm}\]
and
\[\lambda = \frac{2}{r} \approx 0.52.\]

\item To minimize the surface area under the assumptions we have made, the radius should be approximately 3.84 cm and the height approximately 7.66 cm.

\ea

\end{activitySolution}
\aftera